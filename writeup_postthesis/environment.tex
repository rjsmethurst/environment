\documentclass[useAMS,usenatbib]{mn2e}
\usepackage{footnote,graphicx,natbib,color,multirow,amsmath,url}
\usepackage{amssymb}
\usepackage{tabularx}
\usepackage{hyperref}
\usepackage{amssymb}

\hypersetup{
    colorlinks,
    citecolor=black,
    filecolor=black,
    linkcolor=black,
    urlcolor=black
}

\def\lesssim{\mathrel{\hbox{\rlap{\hbox{\lower3pt\hbox{$\sim$}}}\hbox{\raise2pt\hbox{$<$}}}}}


\begin{document}


\title[Group environment quenching mechanisms]{Galaxy Zoo: The interplay of quenching mechanisms in the group environment}
\author[Smethurst et al. 2015]{R. ~J. ~Smethurst,$^{1}$ C. ~J. ~Lintott,$^{1}$ K.~L.~Masters,$^{2}$ R. C. ~Nichol,$^{2}$ \newauthor R. ~Hart,$^{3}$ and the Galaxy Zoo team \footnotemark[1]
\\ $^1$ Oxford Astrophysics, Department of Physics, University of Oxford, Denys Wilkinson Building, Keble Road, Oxford, OX1 3RH, UK 
\\ $^2$ Institute of Cosmology and Gravitation, University of Portsmouth, Dennis Sciama Building, Barnaby Road, Portsmouth, PO1 3FX, UK 
\\ $^3$ School of Physics and Astronomy, The University of Nottingham, University Park, Nottingham, NG7 2RD, UK
}

\maketitle

\begin{abstract}
Does the environment of a galaxy directly influence the quenching history of a galaxy? Here we construct a sample of group and field galaxies with morphological classifications from Galaxy Zoo 2 and use Bayesian inference to determine the quenching time and rate that describes a simple exponentially declining SFH for a given galaxy from its optical and NUV colours. We observe how the detailed morphological structures, such as bars and bulges, are affected by the group environment and correlate this with changes in the SFH. We find that the time since quenching and the rate of quenching do not correlate with the velocity of a satellite through the group but are correlated with the group potential. This quenching occurs within an average quenching timescale of $\sim4~\rm{Gyr}$ from star forming to complete quiescence, during an average infall time of $\sim 3 ~\rm{Gyr}$. This suggests that the environment does play a direct role in galaxy quenching through a quenching mechanism which is correlated with the group potential, such as harassment, interactions and starvation. Environmental quenching mechanisms which are correlated with satellite velocity, such as ram pressure stripping, are therefore not the main cause of quenching in the group environment. Instead an interplay of mergers, mass \& morphological quenching and environment driven quenching mechanisms dependent on the group potential drive the evolution of group galaxies. 

\end{abstract}

\begin{keywords}
galaxies -- environment, galaxies -- photometry, galaxies -- statistics, galaxies -- morphology
\end{keywords}

\footnotetext[1]{This investigation has been made possible by the participation of over 350,000 users in the Galaxy Zoo project. Their contributions are acknowledged at \url{http://authors.galaxyzoo.org}}

\section{Introduction}\label{sec:intro}

Galaxies are often found clustered together in groups \citep{zwicky38, zwicky52, abell58}, all sharing one large dark matter halo (groups with 100 or more galaxies are referred to as clusters; \citealt{bower04}). Conversely some galaxies are found isolated from others in less dense environments (often referred to as the field), either because they are fossil groups \citep[where all members have eventually merged;][]{ponman94, jones00, jones03} or have truly been isolated for their entire lifetimes. This environmental density is found to be correlated with morphology \citep{dressler80, smail97, poggianti99, postman05, Bamford09}, colour \citep{butcher78, pimbblet02}, quenched galaxy fraction \citep{kauffmann03, Baldry06, peng12, darvish16} and SFR \citep{gomez03}. Star forming disc galaxies tend to be located in low-density environments with quiescent elliptical galaxies in more dense environments. This suggests that the environment may drive a galaxy's transition from star forming in the blue cloud to the quiescent red sequence through quenching of star formation. 
 
Although these correlations were originally interpreted as indicating causation, recent evidence from simulations suggests that quenching mechanisms driven by the environment may not be dominant in the galaxy lifecycle \citep{kimm09, kimm11, hirschmann14, wang14, phillips15, emerick16, fillingham16}. Perhaps, instead, the correlation of increased quenched galaxy fractions with environment density is due to a superposition of other possible quenching mechanisms.

There are many theorised mechanisms which can cause quenching. They are often referred to as either internal mechanisms (caused by the galaxy's `nature') or external mechanisms (caused by the way the galaxy is `nurtured'). The properties of a galaxy and its environment are often thought to control which mechanisms will affect a galaxy throughout its lifetime and subsequently affect the morphology. 

\subsection{Internal Quenching Mechanisms}\label{sec:intquench}

\subsubsection{AGN feedback as a quenching mechanism}\label{sec:agnquench}

There are tight correlations between properties of galaxies, such as the bulge mass, total stellar mass \& stellar velocity dispersion, and black hole mass \citep{magorrian98, marconi03, haringrix04}. This implies a co-evolution between the black hole and its host galaxy therefore suggesting that changes in the SFR and structure of a galaxy could also be tied to black hole activity. This is thought to occur via AGN feedback where the output of energetic material and radiation from the black hole is theorised to either heat or expel the gas needed for SF in a galaxy, causing a quench.

AGN feedback was first suggested as a mechanism for regulating star formation due to the results of simulations wherein galaxies could grow to unrealistic stellar masses \citep{silk98, Bower06, Croton06, somerville08}. Without a prescription for the effects of AGN feedback, the shape of the galaxy luminosity function could therefore not be matched at the high luminosity end \citep{baugh98, baugh05, kauffmann99a, kauffmann99b, somerville01, kitzbichler06}. 

Indirect observational evidence has been found for both positive and negative feedback in various systems (see the comprehensive review from \citealt{fabian12}). The strongest being the indirect evidence that the largest AGN fraction is found in the green valley \citep{cowie08, Hickox09, schawinski10a}, suggesting a link between AGN activity and the process which moves a galaxy from the blue cloud to the red sequence. Recent statistical evidence from \cite{smethurst16} has shown the dominance of rapid, recent quenching within a population of Type 2 AGN host galaxies, suggesting that AGN feedback is indeed an important evolutionary mechanism. 


\subsubsection{Mass quenching}\label{sec:massquench}

Mass quenching is defined by \citet{peng10, peng12} as any quenching mechanism acting independently of a galaxy's environment, but not of its mass. However, there is still much debate over the exact mechanism which is the cause of such a quench. \citet{darvish16} suggest that non-AGN driven feedback mechanisms (for example supernova feedback) are responsible for the correlation observed between the mass quenching efficiency and SFR in \citet{peng10}. However, \citet{gabor15} suggest that this is driven by ``halo quenching processes'' whereby the inflow of cool gas from the galaxy halo is either cut off or hindered from cooling at $M_{halo} > 10^{12}~\rm{M}_{\odot}$ \citep{birnboim03, dekel06}. If this happens, a galaxy uses up the rest of its available gas for star formation via the Kennicutt-Schmidt law \citep{schmidt59, kennicutt98} and consequently grows in mass.

In this work, we refer to mass quenching as a cut off of gas inflow, resulting in a gradual consumption of gas in star formation. This definition of mass quenching is thought to be a dominant mechanism for isolated galaxies in the field \citep{kormendy04}. However, it is also thought that as a galaxy infalls in to a group or cluster over long timescales, gas reservoirs can also be depleted via a mass quenching process \citep{peng12}. 

 
\subsubsection{Morphological quenching}\label{sec:morphquench}

Morphological quenching is the process by which the internal structure of a galaxy can have a negative impact on its own SFR. This can happen in one of two ways, either by preventing star formation from occurring or by increasing the rate of consumption of gas for star formation. The former is theorised to be caused by bulges \citep{bluck14} whereby the large gravitational potential of the bulge prevents the disc from collapsing and forming stars \citep{Fang13}. 

The latter mechanism is theorised to occur in galaxies hosting bars; the bar funnels gas to the centre of the galaxy \citep{athanassoula92a} where gas is exhausted by star formation effectively quenching the galaxy \citep{zurita04, sheth05}. This process is thought to be responsible for large numbers of red spirals and supported by observations of increasing bar fraction with red colours \citep{masters11a}. Recent observational evidence from \cite{hart16} also suggests that spiral structure can also cause morphological quenching. \citeauthor{hart16} propose that many armed spiral structures can trigger galaxy wide starbursts thereby rapidly using up gas for future star formation; similarly two armed spirals are observed with redder colours, suggesting that this spiral phase is much longer lived and may funnel gas into the centre of the galaxy to be exhausted in star formation over longer timescales.  
 
\subsection{External Quenching Mechanisms}\label{sec:extquench}

\subsubsection{Mergers as a quenching mechanism}\label{sec:mergersquench}

Major mergers have been intrinsically linked to the formation of elliptical galaxies since \citet{toomre72} showed this was possible with a simulation of the merger of two equal mass disc galaxies. The hypothesis is as follows: when two galaxies merge, the influx of cold gas funnelled by the forces in the interaction often results in energetic starbursts \citep{mihos94, mihos96, hopkins06d, hopkins08a, hopkins08b, snyder11, hayward14, sparre16}, which can exhaust the gas required for star formation, effectively quenching the post-merger remnant. This remnant galaxy will also have formed a dynamically hot bulge through the dissipation of angular momentum in the merger \citep{toomre77, walker96, kormendy04, hopkins11c, martig12}. The mass ratio of the two galaxies merging is thought to affect the size of the bulge that is formed in the remnant \citep{cox08, hopkins09c, tonini16}, with the most massive major mergers with a 1:1 mass ratio producing fully elliptical galaxies \citep{toomre72, barnes96, mihos96, kauffmann96, pontzen16}.

Mergers also have a clear environmental dependence, as they are more likely to occur in denser environments. However, their effects must be separated from those quenching mechanisms driven solely by the properties of the galaxy environment. 

\subsubsection{Environment driven quenching}\label{sec:envquench}

The proposed quenching mechanisms under the umbrella of environmental quenching are numerous and varied. Together with the typical gravitational galaxy-galaxy interactions \citep{moore96} which are expected to be more frequent in a dense environment, environmental quenching also includes hydrodynamic interactions occurring between the cold interstellar medium (ISM) of the in-falling galaxy and the hot intergalactic medium (IGM) of the group or cluster. Such hydrodynamic interactions include ram pressure stripping \citep{gunngott72}, viscous stripping \citep{nulsen82}, and thermal evaporation \citep[a rapid rise in temperature of the ISM due to contact with the IGM;][]{cowie77}. Another such process is starvation \citep[also called strangulation;][]{larson80} which can remove the outer galaxy halo, thus cutting off the star formation gas supply to a galaxy. Preprocessing occurs when all of the above mechanisms take place in a group of galaxies which then merges with a larger group or cluster \citep{dressler04}. 

The most likely (and therefore the most studied) candidate mechanism for the cause of the environmental density-morphology and SFR relations is ram pressure stripping \citep[RPS;][]{abadi99, poggianti99}. However, there has been mounting evidence that RPS can only strip a galaxy of $40-60\%$ of its gas supply \citep{fillingham16} and so may not be as effective a quenching mechanism as first thought \citep{emerick16}. Therefore investigations of other environmentally driven quenching mechanisms, such as strangulation \citep{peng15, hahn16, maier16, paccagnella16, roberts16, vandevoort16} and harassment \citep[high speed galaxy `fly-by' gravitational interactions][]{bialas15, smith15b} are having a recent resurgence. 


\subsection{Aims of this work}\label{sec:aims} 
  
In order to isolate the cause of the density-morphology and density-SFR correlations, we need to observe how morphology and galaxy quenching timescales change in dense environments with different properties in comparison to the field. Here, we consider the group environment (as this is a more typical environment for a galaxy than the relatively rare rich cluster environment \citep{carlberg04}) by constructing a sample of both group and field galaxies and use a Bayesian inference method to determine the quenching time and rate describing a simple exponentially declining SFH for a galaxy given its optical and NUV colours. From these inferred SFHs we aim to constrain the possible mechanisms at work in the group environment. However, dense environments are messy with many possible mechanisms at work, whose effects are difficult to disentangle. 

We aim to determine the following: (i) How does the environment influence the detailed morphological structures of a galaxy?  (ii) Is quenching which is directly caused by the environment occurring in galaxy groups?
 
This paper proceeds as follows. In Section~\ref{sec:data} we describe our data sources and inference methods and highlight our results in Section~\ref{sec:results}. We then discuss the possible quenching mechanisms that could cause our results and how they fit into the bigger picture of quenching in Section~\ref{sec:disc}. We summarise our findings in Section~\ref{sec:conc}. The zero points of all magnitudes are in the AB system. Where necessary, we adopt the WMAP Seven-Year Cosmology \citep{jarosik11} with $(\Omega_m , ~\Omega_\Lambda , ~h) = (0.26, 0.73, 0.71)$.

 
\section{Data and Methods}\label{sec:data}

\subsection{Data Sources}\label{sec:photo}

In this investigation we use visual classifications of galaxy morphologies from the Galaxy Zoo 2\footnote{\url{http://zoo2.galaxyzoo.org/}} (GZ2) citizen science project \citep{GZ2}, which obtains multiple independent classifications for each optical image. The full question tree for an image is shown in Figure~1 of \citeauthor{GZ2}  The GZ2 project used $304, 022$ images from the Sloan Digital Sky Survey Data Release 7 (SDSS; \citealt{york00, abazajian09}) all classified by \emph{at least} 17 independent users, with a mean number of classifications of $\sim42$. We also utilise the Petrosian magnitude, {\tt petroMag}, values for the $u$ ($3,543 \rm{\AA}$) and $r$ ($6,231 \rm{\AA}$) wavebands provided by the SDSS pipeline for the GZ2 galaxies.

Further to this, we required NUV ($2,267 \rm{\AA}$) photometry from the GALEX survey \citep{martin05}, within which $\sim42\%$ of the GZ2 sample was observed, giving $126, 316$ galaxies total ($0.01 < z < 0.25$). This will be referred to as the \textsc{gz2-galex} sample. The completeness of this sample ($-22 < M_u < -15$) is shown in Figure~2 of \cite{smethurst15}. 

Magnitudes are corrected for galactic extinction \citep{Oh11} by applying the \citet{Cardelli89} law, giving a typical correction of $u-r \sim 0.05$. K-corrections are also adopted to $z=0.0$ and absolute magnitudes obtained from the NYU-VAGC \citep{Blanton05, padmanabhan08, blanton07}, giving a typical $u-r$ correction of $\sim 0.15$ mag. The change in the $u-r$ colour due to both corrections therefore ranges from $\Delta (u-r) \sim 0.2$ at low redshift, increasing up to $\Delta (u-r) \sim 1.0$ at $z \sim 0.25$, which is consistent with the expected k-corrections shown in Figure 15 of \citet{blanton07}. These corrections were calculated by \citet{Bamford09} for a subset of galaxies in the SDSS survey.

Galaxy stellar masses are estimated using the method outlined in \cite{Baldry06}, who fit a relationship between the observed $u-r$ colour and the $r$-band mass-to-light ratio, $(M_*/L_r)$, of a galaxy. The $r$-band absolute magnitude, $\mathcal{M}_r$, of a galaxy can be used to estimate $(M_*/L_r)$, as outlined in \cite{blanton01}. 


\subsection{Group Identification}\label{sec:groups}

The construction of a robust cluster or group catalogue is a challenge, with many studies attempting this across the SDSS \citep{merchan05, miller05, berlind06, yang07, tago08, tago10, tinker11, munoz12, tempel14} and other large surveys \citep{tucker00, merchan02, eke04, cucciati10, robotham11, knobel12}. The difficulties arise in removing projection effects, understanding the selection function used, covering large ranges in mass and redshift, and dealing with spectral fibre collisions (see the comprehensive review by \citet{postman02} for an in depth discussion). 

The \citeauthor{yang07} catalogue is the most commonly used in environment studies using data from the SDSS \citep[including][]{hoyle11, pasquali12, wetzel14, shankar14, lacerna14, knobel15, fitzpatrick15, lan16, woo16, bluck16, weigel16}, but I find that when cross matched with the \textsc{gz2-galex} sample (with a $3''$ search radius) only $38$ galaxies (of $176,604$ possible galaxies) which belong to a group with 2 or more members are identified. This is most likely due to the necessity for GALEX NUV photometry in this study. The majority of NUV emission comes from massive, short lived stars and so in the cluster environment which is dominated by typical `red and dead' quiescent galaxies, detecting NUV emission is less likely. 

In this work we use the \citet{berlind06} catalogue, which was cross matched with the \textsc{gz2-galex} sample. We limited the sample to $z < 0.1$ to ensure GALEX completeness to the red sequence, as in \citealt{wyder07, yesuf14}, so that we do not introduce any bias in our sample due to the necessity for NUV colours. This results in $14,199$ group galaxies with the number of group members, $N_{\rm{group}} \geq 2$. Centrals were selected as the most massive galaxy in a group \citep[as in][]{yang07, yang09, pasquali10} with all other galaxies in a group designated as satellites.

\begin{figure*}
\centering
\includegraphics[width=\textwidth]{sfr_mass_quenched_centrals_satellites_gz2_group.pdf}
\caption[Stellar mass-SFR plane for the centrals and satellites of the \textsc{gz2-group} sample]{The stellar mass-SFR plane showing central (left; red contours) and satellite (right; blue contours) in the \textsc{gz2-group} sample. In both panels the entire SDSS sample from the MPA-JHU catalogue is shown by the grey contours. The definition of the SFS from \cite{peng10} at $\overline{z} = 0.053$ (solid line, the mean redshift of the \textsc{gz2-group} sample) with $\pm1\sigma$ (dashed lines) is shown.}
%KS Test between distributions?
\label{fig:sfrmass}
\end{figure*}


The projected group-centric radius, $R$, of all satellite galaxies was calculated and converted to $\rm{kpc}$ by using the observed spectroscopic redshift of the central galaxy. In order to compare groups of different sizes, the virial radius is used as a normalisation factor to this projected group-centric radius. Here we use a proxy to the virial radius, $R_{200}$ \citep[see][]{navarro95}, the radius within which the group mass overdensity is 200 times the critical density, $\rho_{\rm{crit}}(z)$, as defined by \citealt{finn05}:
\begin{equation}\label{eq:overdense}
200\rho_{\rm{crit}}(z) = \frac{M_{cl}}{\frac{4}{3}\pi R_{200}^3},
\end{equation}

where $M_{cl}$ is the mass of the group. \citeauthor{finn05} then use the redshift dependence of the critical density and the virial mass to relate the line-of-sight velocity dispersion, $\sigma_x$, to the group mass so that $R_{200}$ becomes:
\begin{equation}\label{eq:r200}
R_{200} = 1.73 \left ( \frac{\sigma_x}{1000 \rm{km}~\rm{s}^{-1}} \right) \cdot \frac{1}{\sqrt{\Omega_{\Lambda} +\Omega_o(1+z)^3}} ~ h_{100}^{-1} ~\rm{Mpc}, 
\end{equation}

$\sigma_x$ is calculated for a group as the standard deviation of the velocity dispersions $\sqrt{(v_i - \left< v_i\right>)^2}$. Here $v_i$ are the proper velocities of each galaxy, $i$, as defined in \cite{danese80}:
\begin{equation}\label{eq:propervel}
v_i = c \cdot \frac{z_i - z_{group}}{1 + z_{group}},
\end{equation}
where $z_{group}$ is the mean redshift of all the group members. Since most groups in the sample have low $N_{\rm{group}}$, using the mean redshift for $z_{\rm{group}}$, rather than the central galaxy redshift is most appropriate in this case. These calculations resulted in a sample of $3,468$ centrals and $10,731$ satellites within a projected group-centric radius range of $0.02 < R/R_{200} < 24.9$ and $z < 0.084$ which shall be referred to as the \textsc{gz2-berlind} sample. Note that for a galaxy (central or satellite) to be included in the \textsc{gz2-berlind} sample, the rest of its group does not. However the properties of that group are still retained by the included galaxy. 

We obtain SFRs and stellar velocity dispersions of galaxies in the \textsc{gz2-berlind} sample from the MPA-JHU catalogue \citep{kauffmann03, brinchmann04}. In this study we specifically focus on galaxies that are below the star forming sequence (SFS). We select a subsample of the \textsc{gz2-berlind} galaxies that are $1\sigma$ below the SFS (as defined by \cite{peng10}), giving $4,629$ satellite and $2,314$ central galaxies which will collectively be referred to as the \textsc{gz2-group} sample. These galaxies are shown in the panels of Figure \ref{fig:sfrmass} and can be seen to lie below the SFS.



We also compare the \textsc{gz2-berlind} and \textsc{gz2-group} samples with a measurement of the projected neighbour density from \cite{Baldry06}, $\Sigma_N = N/4\pi d_N^2$, where $d_N$ is the distance to the $N^{\rm{th}}$ nearest neighbour. $\Sigma$ is a more direct probe of the local density of a galaxy's environment, and although it does not allow for the identification of groups and their properties, it is still a useful probe of the local density inside a group  \cite[see][for a comparison of various environment parametrisations]{muldrew12}.

In this work we use the estimates of \cite{Bamford09} who calculated the local galaxy density, $\Sigma$, determined by averaging $\log\Sigma_N$ for $N = 4$ and $N=5$ by the method outlined in \citet{Baldry06}, for the entirety of the GZ2 sample. $90\%$ of the \textsc{gz2-berlind} sample have $\log\Sigma > -0.8$ (the threshold quoted by \citealt{Baldry06} to define non-field galaxies), suggesting a purity of $\sim90\%$ for the \textsc{gz2-berlind} sample.% The distributions of $\log\Sigma$ for star forming and quenching/quenched centrals and satellites in the \textsc{gz2-berlind} sample are shown in Figure~\ref{fig:sigmadist}. Star forming galaxies tend to reside in less dense local environments than their quenching or quenched counterparts. The satellite galaxies as a whole also seem to occupy denser local environments than centrals, however on investigation this seems to arise because the satellites in the \textsc{gz2-berlind} sample reside in groups with larger $N_{group}$ than the centrals. This is to be expected given the definition of a satellite galaxy. 

% \begin{figure*}
% \centering
% \includegraphics[width=\textwidth]{SIGMA_density_sf_q_cent_sat.pdf}
% \caption[Local environment density distributions of central and satellite galaxies]{Local environment density, $\log\Sigma$, distributions of star forming (black) and quenching/quenched (red) central (left) and satellite (right) galaxies in the \textsc{gz2-group} sample.}
% %KS Test between distributions?
% \label{fig:sigmadist}
% \end{figure*}


\subsection{Field sample}\label{sec:field}


We constructed a sample of field galaxies for use as a control sample to the \textsc{gz2-group} sample. For all galaxies in the \textsc{gz2-galex} sample, we calculated the smallest projected group-centric radii, $R/R_{200}$, from each of the central galaxies in the \citet{berlind06} catalogue (regardless of whether the central was included in the \textsc{gz2-berlind} sample). We also utilise the measurement of the projected neighbour density, $\Sigma$, from \cite{Baldry06}. We select candidate field galaxies as those with (i) $R/R_{200} > 25$ and (ii) $\log\Sigma < -0.8$ \citep[the threshold on the local environment density which selects field galaxies as defined by][]{Baldry06}. We chose to use both of these environmental density measures to ensure a pure sample of candidate field galaxies.

This sample of field galaxy candidates was then matched in redshift and stellar mass firstly to the central galaxies of the \textsc{gz2-group} sample to give $2,309$ field galaxies with $z < 0.084$. In this study we focus on galaxies which are either quenching or quenched and are more than $1\sigma$ below the SFS and so the same constraints must be placed on this field control sample. This encompasses $1,596$ field galaxies with $z < 0.084$ which will be referred to as the \textsc{gz2-cent-field-q} sample. It will be used as a control sample when investigating the trends with central galaxy properties of the inferred quenching parameters. The redshift distribution of the \textsc{gz2-cent-field-q} sample is shown in comparison to the distribution of central galaxies in the \textsc{gz2-group} sample in the left panel of Figure~\ref{fig:zcompare}. %SDSS images of a random selection of galaxies from the \textsc{gz2-group} and \textsc{gz2-cent-field-q} samples are shown ordered by their GZ2 debiased vote fraction in Figure~\ref{fig:mosaic}. %KS test between samples?

Secondly, the field galaxy candidates were matched in redshift and stellar mass to the satellite galaxies of the \textsc{gz2-group} sample to give $5, 004$ field galaxies with $z < 0.084$ which will be referred to as the \textsc{gz2-sat-field} sample. These galaxies will be used as a control when investigating the morphological trends of satellite galaxies with environment. Note that this sample is not restricted to being $1\sigma$ below the SFS. $237$ galaxies are found in both the \textsc{gz2-cent-field-q} and \textsc{gz2-sat-field} samples. The redshift distribution of the \textsc{gz2-sat-field} sample is shown in comparison to the distribution of satellite galaxies in the \textsc{gz-group} sample in the right panel of Figure~\ref{fig:zcompare}.

We once again obtain SFRs and stellar velocity dispersions of galaxies for all of the field samples described above from the MPA-JHU catalogue \citep{kauffmann03, brinchmann04}.

\begin{figure}
\centering{
\includegraphics[width=0.45\textwidth]{redshift_cent_field.pdf}
\includegraphics[width=0.45\textwidth]{redshift_sat_field.pdf}}
\caption[Redshift distribution of galaxies in the \textsc{gz2-group} sample]{Redshift distributions of central (left) and satellite galaxies (right) in the \textsc{gz2-group} sample (black solid line) in comparison the redshift matched \textsc{gz2-cent-field-q} (left; blue dashed line) and \textsc{gz2-sat-field} samples (right; blue dashed line).}
%KS Test between distributions?
\label{fig:zcompare}
\end{figure}

\subsection{Morphological fractions}\label{sec:morphfrac}

We utilise the GZ2 vote fractions to quantify the morphology of galaxies in the \textsc{gz2-group} sample, in order to investigate the morphological trends with group radius. We utilise $p_{\rm{disc}}$ and $p_{\rm{smooth}}$ to characterise the likelihood of galaxies being either discs or ellipticals. We also use vote fractions from further down the GZ2 decision tree including $p_{\rm{bar}}$, $p_{\rm{bulge}}$ and $p_{\rm{merger}}$ to calculate the bar, bulge and merger fractions in the \textsc{gz2-group} sample respectively. 

Fractions are calculated considering the number of barred (with $p_{\rm{bar}} > 0.5$; see \citealt{masters11a, Cheung13}) and bulged (with $p_{\rm{obvious}~\rm{or}~\rm{dominant}} > 0.5$ and $p_{\rm{none}~\rm{or}~\rm{noticeable}} > 0.5$) galaxies over the number of disc galaxies ($p_{\rm{disc}} > 0.43$, $p_{\rm{edge\_on, no}} > 0.715$, $N_{\rm{edge\_on, no}} > 20$; see Table 3 of \citealt{GZ2} for appropriate thresholds on the GZ2 vote fractions to select a sample of galaxies with a particular morphology) in the \textsc{gz2-group} satellite sample. The merger fraction considers the number of merging galaxies (with $p_{\rm{merger}} > 0.4$; see \citealt{Darg10a}) over the number of galaxies in the \textsc{gz2-group} satellite sample. 


\subsection{Deriving quenching parameters}\label{sec:starpy}

\textsc{starpy}\footnote{Publicly available: \url{http://github.com/zooniverse/starpy}} is a \textsc{python} code which allows the user to derive the quenching star formation history (SFH) of a single galaxy through a Bayesian Markov Chain Monte Carlo method \citep{emcee13}\footnote{\url{http://dan.iel.fm/emcee/}} with the input of the observed $u-r$ and $NUV-u$ colours, a redshift, and the use of the stellar population models of \cite{BC03}.  These models are implemented using solar metallicity (varying this does not substantially affect these results; \citealt{smethurst15}) and a Chabrier IMF \citep{chabrier03} but does not model for intrinsic dust. The SFH is modelled as an exponential decline of the SFR described by two parameters $[t_{q}, \tau]$, where $t_{q}$ is the time at the onset of quenching $\rm{[Gyr]}$ and $\tau$ is the exponential rate at which quenching occurs $\rm{[Gyr]}$. Under the simplifying assumption that all galaxies formed at $t=0$ $\rm{ Gyr}$ with an initial burst of star formation, the SFH can be described as: 
\begin{equation}\label{sfh}
SFR =
\begin{cases}
i_{sfr}(t_{q}) & \text{if } t < t_{q} \\
i_{sfr}(t_{q}) \times exp{\left( \frac{-(t-t_{q})}{\tau}\right)} & \text{if } t > t_{q} 
\end{cases}
\end{equation}
where $i_{sfr}$ is the constant star formation rate (SFR) defined so that at the time of quenching, $t_{q}$, the modelled galaxy resides on the `star forming sequence' (SFS). We use the definition of the SFS from \cite{peng10} for a galaxy with mass, $m = 10^{10.27} M_{\odot}$ (the mean mass of the \textsc{gz2-galex} sample) at the redshift of the observed galaxy.  A smaller $\tau$ value corresponds to a rapid quench, whereas a larger $\tau$ value corresponds to a slower quench. We note that a galaxy undergoing a slow quench is not necessarily quiescent by the time of observation. Similarly, despite a rapid quenching rate, star formation in a galaxy may still be ongoing at very low rates, rather than being fully quenched. This SFH model has previously been shown to appropriately characterise populations of quenching or quiescent galaxies \citep{Weiner06, Martin07, Noeske07,schawinski14}. We note also that star forming galaxies in this regime are fit by a constant SFR with a $t_{q} \simeq$ Age$(z)$, (i.e. the age of the Universe at the galaxy's observed redshift) with a very low probability.

% \begin{figure}
% \centering{
% \includegraphics[width=0.45\textwidth]
% {triangle_t_tau_red_s_1237655504035185152_40000_14_16_06_08_14.pdf}}
% \caption{Example output from \textsc{starpy} for a single galaxy. The contours show the positions of the `walkers' in the Markov Chain (which are analogous to the areas of high probability) for the SFH models described by $\theta = [t_q, \tau]$. The histograms show the 1D projection along each axis. Solid (dashed) blue lines show the best fit parameters (with $\pm 1\sigma$) to the data. The postage stamp image from SDSS is shown in the top right along with the debiased vote fractions for smooth ($p_s$) and disc ($p_d$) from GZ2.}
% \label{fig:one_example}
% \end{figure}

The probabilistic fitting methods to these star formation histories for an observed galaxy are described in full detail in Section 3.2 of \cite{smethurst15}, wherein the \textsc{starpy} code was used to characterise the SFHs of each galaxy in the \textsc{gz2-galex} sample. We assume a flat prior on all the model parameters and the difference between the observed and predicted $u-r$ and $NUV-u$ colours are modelled as independent realisations of a double Gaussian likelihood function (Equation 2 in \citealt{smethurst15}). We also make the simplifying assumption that the age of each galaxy, $t_\mathrm{age}$ corresponds to the age of the Universe at its observed redshift, $t_\mathrm{obs}$. An example posterior probability distribution output by \textsc{starpy} is shown for a single galaxy in Figure 5 in \cite{smethurst15}, wherein the degeneracies of the SFH model can be clearly seen. These degeneracies are present for all galaxies run through \textsc{starpy} therefore if differences in the distributions arise when comparing two galaxies (or two populations), this is due to intrinsic differences in their SFHs and not due to the degeneracies of the model. The best fit $[t_q, \tau]$ parameters for a single galaxy are estimated from the median of the posterior probability distribution output by \textsc{starpy} (the 50th percentile position of the MCMC chain, with the $\pm1\sigma$ derived from the 16th and 84th percentile positions, see Section 3.2 of \citealt{smethurst15})

We note also that galaxy colours were not corrected for intrinsic dust attenuation. This is of particular consequence for disc galaxies, where attenuation increases with increasing inclination. \cite{Buat05} found the median value of the attenuation in the GALEX NUV passband to be $\sim 1$ mag. Similarly \cite{masters10c} found a total extinction from face-on to edge-on spirals of 0.7 and 0.5 mag for the SDSS $u$ and $r$ passbands and show spirals with $\log(a/b) > 0.7$ have signs of significant dust attenuation. However, we showed from an investigation into this problem in Section~2.2 of \citet{smethurst16} that internal galactic extinction does not systematically bias our results from \textsc{starpy}. 

\section{Results}\label{sec:results}

We now investigate the environmental dependence of detailed morphological structures in the group environment and compare how these correlate with the environmental dependence of the inferred quenching histories of galaxies in the \textsc{gz2-group} sample. 

\subsection{Mass dependence with radius}

Since morphological features have been shown to be dependent on the stellar mass of a galaxy \citep[e.g. the increase in the bar fraction with stellar mass; see][]{nair10, skibba12}, before investigating trends in the morphology with group radius in the \textsc{gz2-group} sample, the mass dependence on the group radius must be considered. This is shown in Figure~\ref{fig:massdep}. The mean stellar mass is roughly flat and consistent with the median field value with increasing group radius, until the most central group radius bin at $R \sim 0.1~R_{200}$. This trend is present for both morphologies, with early-type galaxies showing a larger increase in the average stellar mass. We note that if this inner bin at $0.1 R/R_{200}$ is ignored in the results that follow, the conclusions still hold. 

\begin{figure}
\centering{
\includegraphics[width=0.45\textwidth]{mass_trend_with_log_radius_compare_field.pdf}}
\caption[Average mass with group radius in the \textsc{gz2-group} sample]{The average stellar mass as a function of radius from the group centre. The shaded regions show the $\pm1\sigma$ in each bin of $R/R_{200}$. The average stellar mass of the \textsc{gz2-sat-field} sample is also shown (blue solid line) with $\pm1\sigma$ (blue dashed line).}
\label{fig:massdep}
\end{figure}



\subsection{Dependence of detailed morphological structure with environment}\label{sec:resmorph}

\begin{figure*}
\includegraphics[width=0.43\textwidth]{p_disc_trend_with_log_radius_field_compare.pdf}
\includegraphics[width=0.43\textwidth]{p_smooth_trend_with_log_radius_field_compare.pdf}
\caption[Mean $p_d$ and $p_s$ with group radius in the \textsc{gz2-group} sample]{Mean GZ2 vote fraction for disc (left) and smooth (right) galaxies in the \textsc{gz2-group} sample binned by projected group-centric radius, normalised by $R_{200}$, a proxy for the virial radius of a group. The shaded region shows $\pm1\sigma$ on the mean vote fraction. The mean vote fraction of the \textsc{gz2-sat-field} sample are also shown (blue solid lines) with $\pm1\sigma$ (blue dashed lines).}
\label{fig:morphradius}
\end{figure*}

\begin{figure}
\centering{
\includegraphics[width=0.4\textwidth]{bar_fraction_over_disc_trend_with_log_radius_sat_matched_field_cand.pdf}}
\caption[Bar fraction with group radius in the \textsc{gz2-group} sample]{Bar fraction (number of barred disc galaxies over number of disc galaxies) in the \textsc{gz2-group} sample binned in projected group-centric radius, normalised by $R_{200}$, a proxy for the virial radius of a group. The shaded region shows $\pm1\sigma$ on the bar fraction. The bar fraction of the \textsc{gz2-sat-field} sample is also shown (blue solid line) with $\pm1\sigma$ (blue dashed line).}
\label{fig:barradius}
\end{figure}

\begin{figure}
\centering{
\includegraphics[width=0.43\textwidth]{merger_fraction_trend_with_log_radius_compare_sat_field_cand.pdf}}
\caption[Merger fraction with group radius in the \textsc{gz2-group} sample]{Merger fraction in the \textsc{gz2-group} sample binned in projected group-centric radius, normalised by $R_{200}$, a proxy for the virial radius of a group. The shaded region shows $\pm1\sigma$ on the merger fraction. The merger fraction of the \textsc{gz2-sat-field} sample is also shown (blue solid line) with $\pm1\sigma$ (blue dashed line).}
\label{fig:mergerradius}
\end{figure}


We perform an initial sanity check on the \textsc{gz2-group} sample by recreating the morphology-density relation of \citet{dressler80} in Figure \ref{fig:morphradius}, which shows the mean disc and smooth vote fractions as a function of group radius. The mean disc vote fraction decreases from the mean field value (blue line) within $1$ virial radius. Simultaneously, the mean smooth vote fraction increases, which is in agreement with previous studies on the morphology-density relation \citep{dressler80, smail97, poggianti99, postman05, Bamford09}. The extensive morphological classifications provided by GZ2 also allow for the investigation of how more detailed morphological structure is affected by the group environment.  

Figure \ref{fig:barradius} shows how the bar fraction (number of barred disc galaxies over the number of disc galaxies; see Section~\ref{sec:morphfrac}) increases significantly over the field fraction (blue solid line) with decreasing group-centric radius, in agreement with the findings of \cite{barazza09}. However, we also observe a dip in the bar fraction in the most central projected group centric radius bin in agreement with \cite{skibba12}. 

In Figure \ref{fig:mergerradius} we show how the merger fraction does not significantly deviate from the field fraction (blue solid line) except for galaxies found within one virial radius. As discussed in Section~\ref{sec:mergersquench}, mergers are thought to drive bulge growth and so similarly, Figure~\ref{fig:bulgeradius} shows how the fraction of galaxies with obvious/dominant bulges increases over the field value in the inner regions of the group (in agreement with \citealt{diaferio01}) and the fraction of those with none/just noticeable bulges decreases below the field value within $1$ virial radius. 

%Figure \ref{fig:sfrradius} shows how the SFR of the \textsc{gz2-group} sample does indeed decline with decreasing group-centric distance, significantly below the mean SFR of the \textsc{gz2-field} sample shown by the blue dashed line. This is in agreement with the results of \cite{gomez03} who observe a similar decline in SFR with group-centric radius in SDSS clusters (see for example, Figure 6 in \citealt{gomez03}). This coincides with the morphological fraction changes seen in Figures~\ref{fig:morphradius}-{\ref{fig:merger radius} in support of the conclusions of \citet{smethurst15} that quenching is morphologically dependent. 


%\begin{figure}
%\includegraphics[width=0.46\textwidth]{sfr_trend_with_log_radius_field_matched_blue_dashed_hlines_gomez_03_rv_not_r200.pdf}
%\caption{Median $H\alpha$ derived star formation rates of satellite galaxies in the \textsc{gz2-group} sample, binned in projected group-centric radius, normalised by $R_{200}$, a proxy for the virial radius of a group.  The shaded region shows the SFRs encompassed by $50\%$ of the population in a given bin. The median SFR of the \textsc{gz2-sat-field} sample is shown (blue solid line) along with the 25th and 75th percentiles (blue dashed lines).}
%\label{fig:sfrradius}
%\end{figure}

\subsection{Quenching histories in the group environment}\label{sec:resultssfhs}

\begin{figure*}
\centering{
\includegraphics[width=0.95\textwidth]{min_max_bulge_fraction_trend_with_log_radius_sat_field_cand.pdf}}
\caption[Bulge fraction with group radius in the \textsc{gz2-group} sample]{Fraction of galaxies with none/just noticeable bulge classifications (left) and with obvious/dominant bulge classifications (right) in the \textsc{gz2-group} sample binned in projected group-centric radius, normalised by $R_{200}$, a proxy for the virial radius of a group. The shaded regions shows $\pm1\sigma$ on the bulge fractions. The bulge fractions of the \textsc{gz2-sat-field} sample are also shown (blue solid lines) with $\pm1\sigma$ (blue dashed lines).}
\label{fig:bulgeradius}
\end{figure*}

The SFHs of all galaxies in both the \textsc{gz2-group} and \textsc{gz2-cent-field-q} samples were analysed using \textsc{starpy}, providing the posterior probability distribution across the two-parameter space for an individual galaxy. In \cite{smethurst15} and \cite{smethurst16} the individual SFHs of the entire \textsc{gz2-galex} sample and those hosting Type 2 AGN, respectively, were combined and weighted to give an overall distribution of the quenching parameters within a population of galaxies. However, in this study, we take the median value (the 50th percentile walker position, see Section~\ref{sec:starpy}) of an individual posterior probability distribution to give the most likely quenching time, $t_{q}$, and quenching rate, $\tau$, for each galaxy. This decision was made because no differences could be seen across the combined population distributions in just three bins of projected group-centric radius.

This simplifies the output from \textsc{starpy} for each galaxy from a probability distribution to just two values, with $\pm1\sigma$ uncertainties, which encompass the spread of the individual galaxy's SFH posterior probability distribution. We then calculate the time since quenching onset, $\Delta t$, for a given galaxy by calculating {\bf $\Delta t = t^\mathrm{obs} - t_{q}$} (where $t^{\rm{obs}}$ is the age of the Universe at a galaxy's observed redshift; see Section~\ref{sec:starpy}). 

With the output from \textsc{starpy} we can now observe the trends in the time since quenching onset, $\Delta t$, and quenching rate, $\tau$, with group radius, $R/R_{200}$, for satellite galaxies and central galaxies in the \textsc{gz2-group} sample, compared with galaxies in the \textsc{gz2-cent-field-q} sample. This is shown in Figures \ref{fig:timesinceradius} - \ref{fig:timesinceradiusvel} wherein the \textsc{gz2-group} galaxies are binned by stellar mass (Figures~\ref{fig:timesinceradius}a-b), halo mass (Figures~\ref{fig:timesinceradius}c-d), mass ratio (Figures~\ref{fig:timesinceradiusmu}a-b), number of group galaxies (Figures~\ref{fig:timesinceradiusmu}c-d), relative velocity (Figures~\ref{fig:timesinceradiusvel}a-b) and stellar velocity dispersion (Figures~\ref{fig:timesinceradiusvel}c-d). All bin thresholds were chosen to give approximately the same number of galaxies in each bin. 


\begin{figure*}
\centering{
\includegraphics[width=0.43\textwidth]{time_since_quenching_M*_Mh.pdf}
\includegraphics[width=0.43\textwidth]{rate_of_quenching_M*_Mh.pdf}
\caption[Trend of $\Delta t$ and $\tau$ with group radius split by stellar mass and halo mass]{The time since quenching onset ($\Delta t = t_{obs} - t_{q}$; left) and rate of quenching ($\tau$; right) binned in group radius, $R/R_{200}$, for satellite galaxies (crosses) split into bins of stellar mass (top) and stellar mass of the corresponding central galaxy (bottom; a proxy for halo mass of a group). The corresponding values for central galaxies (squares, plotted at $\sim0.01 R/R_{200}$) and galaxies in the \textsc{gz2-cent-field-q} sample (circles, plotted at $25 R/R_{200}$) are shown and connected by the dashed lines to help guide the eye. The shaded regions show the $\pm1\sigma$ on $\Delta t$ and $\tau$ in each bin of $R/R_{200}$.}
\label{fig:timesinceradius}}
\end{figure*}

\begin{figure*}
\centering{
\includegraphics[width=0.43\textwidth]{time_since_quenching_mu_Ngroup.pdf}
\includegraphics[width=0.43\textwidth]{rate_of_quenching_mu_Ngroup.pdf}
\caption[Trend of $\Delta t$ and $\tau$ with group radius split by number in group and stellar mass ratio]{The time since quenching onset ($\Delta t = t_{obs} - t_{q}$) and rate of quenching ($\tau$; right) binned in group radius, $R/R_{200}$, for satellite galaxies (crosses) split into bins of stellar mass ratio ($\mu_* = M_*/M_{cent,*}$, top) and number of group members ($N_{group}$, bottom). The corresponding values for central galaxies (squares, plotted at $\sim0.01 R/R_{200}$) and galaxies in the \textsc{gz2-cent-field-q} sample (circles, plotted at $25 R/R_{200}$) are shown, where possible, and connected by the dashed lines to help guide the eye. The shaded regions show the $\pm1\sigma$ on $\Delta t$ and $\tau$ in each bin of $R/R_{200}$.}
\label{fig:timesinceradiusmu}}
\end{figure*}

\begin{figure*}
\centering{
\includegraphics[width=0.43\textwidth]{time_since_quenching_delv_sigma.pdf}
\includegraphics[width=0.43\textwidth]{rate_of_quenching_delv_sigma.pdf}
\caption[Trend of $\Delta t$ and $\tau$ with group radius split by relative velocity and stellar velocity dispersion]{The time since quenching onset ($\Delta t = t_{obs} - t_{q}$; left) and rate of quenching ($\tau$; right) binned in group radius, $R/R_{200}$, for satellite galaxies (crosses) split by the absolute relative velocity of the satellite to its central galaxy ($|\Delta v|$, top) and stellar velocity dispersion ($\sigma_*$, bottom). The corresponding values for central galaxies (squares, plotted at $\sim0.01 R/R_{200}$) and galaxies in the \textsc{gz2-cent-field-q} sample (circles, plotted at $25 R/R_{200}$) are shown, where possible, and connected to the satellite values by the dashed lines to help guide the eye. The shaded regions show the $\pm1\sigma$ on $\Delta t$ and $\tau$ in each bin of $R/R_{200}$.}
\label{fig:timesinceradiusvel}}
\end{figure*}

Across all the left panels in Figures~\ref{fig:timesinceradius} - \ref{fig:timesinceradiusvel} a general trend for increasing time since quenching onset with decreasing group radius can be seen. As in Figures \ref{fig:morphradius}$-$\ref{fig:bulgeradius} significant differences from the mean field values arise at radii less than one virial radius. However, no trend with group radius is seen for the rate at which quenching occurs for satellites in the \textsc{gz2-group} sample (right panels Figures~\ref{fig:timesinceradius} - \ref{fig:timesinceradiusvel}). This suggests that whatever mechanisms cause quenching in a group will do so at the same rate in both the dense inner and sparse outer regions. 

In Figure~\ref{fig:timesinceradius}a the \textsc{gz2-group} sample is split by stellar mass, $M_*$, and a clear trend for increasing $\Delta t$ with increasing stellar mass for satellite, central and field galaxies can be seen. However, this trend is less apparent for the rate of quenching seen in Figure~\ref{fig:timesinceradius}b. The central galaxies (shown by the square points) appear to have quenched more recently than the inner satellites (at $\sim0.1R/R_{200}$) of the same mass but have done so at the same quenching rate. 

In the bottom panels of Figure \ref{fig:timesinceradius} we split the \textsc{gz2-group} sample by halo mass by using the stellar mass of the corresponding central galaxy of a group, $M_{cent,*}$, as a proxy. We find a clear trend for increasing time since quenching onset with increasing halo mass for satellite, central and field galaxies (Figure~\ref{fig:timesinceradius}c) but once again this trend is less apparent for the rate of quenching (Figure~\ref{fig:timesinceradius}d) suggesting that the halo mass does not affect which quenching mechanism acts upon either central or satellite galaxies. 

To account for the effects of conformity, whereby satellites of higher mass tend to be found in higher mass halos \citep{weinmann06, kauffmann13, hearin15, hatfield16}, we also split the satellites of the \textsc{gz2-group} sample by the stellar mass ratio of the satellite to its central galaxy, $\mu_* = M_*/M_{cent,*}$, in the top panels of Figure~\ref{fig:timesinceradiusmu}. $\Delta t $ increases more steeply with group radius (particularly within $\sim$ one virial radius; Figure~\ref{fig:timesinceradiusmu}a) for satellite galaxies with much smaller masses than their group central ($-2.0 < \log_{10}\mu_* < -0.75$, shown by the blue curve). Once again this is not the case for the rate that quenching occurs, as shown in Figure~\ref{fig:timesinceradiusmu}b. 

Another property of the group which is expected to affect the satellite quenching histories is the number of group members, $N_{group}$, which should be roughly correlated with a satellite's local density in a  group. The bottom panels of Figure~\ref{fig:timesinceradiusmu} show that there is no trend with time since quenching onset or rate of quenching with increasing $N_{group}$ for satellite galaxies. The central galaxies (shown by the square points) however, do show a trend for increasing time since quenching as the number of group galaxies increases (Figure~\ref{fig:timesinceradiusmu}c), but the rate at which they quench is the same (Figure~\ref{fig:timesinceradiusmu}d) suggesting the mechanism by which this occurs is the same for all centrals regardless of halo mass. 

In the top panels of Figure \ref{fig:timesinceradiusvel} the \textsc{gz2-group} satellite galaxies are split into bins of their relative velocity to their central galaxies, i.e. the velocity at which they move through the dense group environment. There is no trend with either time since onset of quenching (Figure \ref{fig:timesinceradiusvel}a) or rate of quenching (Figure \ref{fig:timesinceradiusvel}b) with increasing relative velocity for galaxies in the \textsc{gz2-group} sample. This suggests that whatever quenching mechanism is occurring in groups, it is not correlated with the velocity at which satellites move through the dense environment.

The bottom panels of Figure~\ref{fig:timesinceradiusvel} show the trend with group radius for the \textsc{gz2-group} satellites when split into bins of galaxy stellar velocity dispersion $\sigma_*$ (note that this is not the velocity dispersion of the group) which is often used as a proxy for the galaxy potential. The stellar velocity dispersion shows the largest trend in $\Delta t$ (Figure~\ref{fig:timesinceradiusvel}c) for satellite galaxies across all right panels of Figures~\ref{fig:timesinceradius}-\ref{fig:timesinceradiusvel}, with galaxies with the smallest stellar velocity dispersions having quenched more recently. Although this trend is less apparent for the rate that quenching occurs when the satellite galaxies are split by $\sigma_*$ (Figure~\ref{fig:timesinceradiusvel}d), it is the largest trend seen across the right panels of Figures~\ref{fig:timesinceradius}-\ref{fig:timesinceradiusvel}. Also, field galaxies (shown by the circles at $\sim 25 R/R_{200}$) with low velocity dispersions are seen to quench at much slower rates than their satellite counterparts. This suggests that the rapid quenching observed for the low stellar velocity dispersion satellites is directly caused by the environment. 

The results shown in Figures~\ref{fig:timesinceradius}-\ref{fig:timesinceradiusvel} are summarised in Table~\ref{table:resultsum}.

\begin{table*}
\centering
\caption{Summary of results shown in Figures~\ref{fig:timesinceradius}-\ref{fig:timesinceradiusvel} denoting whether there is, $\checkmark$, or isn't, $\times$, a trend with $\Delta t$ when the \textsc{gz2-group} satellite galaxies are split by the stated property.}
\label{table:resultsum}
\begin{tabular*}{\textwidth}{l@{\extracolsep{\fill}}|cccccc}
\hline
\multicolumn{1}{r|}{}   & $M_*$    & $M_{\rm{cent},*}$ & $\mu_*$  & $N_{\rm{group}}$ & $|\Delta v|$ & $\sigma_*$ \\ \hline
Shown in Figure & \ref{fig:timesinceradius}a & \ref{fig:timesinceradius}c & \ref{fig:timesinceradiusmu}a & \ref{fig:timesinceradiusmu}c & \ref{fig:timesinceradiusvel}a & \ref{fig:timesinceradiusvel}c \\
Trend with $\Delta t$ when split by ...?~ & $\checkmark$ & $\checkmark$          & $\checkmark$ & $\times$         & $\times$     & $\checkmark$   \\ \hline
\end{tabular*}
\end{table*}

\section{Discussion}\label{sec:disc}

We shall now consider the results presented in Section~\ref{sec:results} in the context of possible quenching mechanisms which could be responsible. 

\subsection{The role of mergers as quenching mechanisms in the group environment}\label{sec:rolemergerenv}

The merger classification in GZ2 has been shown to preferentially identify major mergers \citep{Darg10a}; while bulge formation in disc galaxies is often associated with evolutionary histories driven by minor mergers \citep{Croton06, tonini16}.  Although we see evidence for an enhanced merger fraction in the inner regions of the group environment in Figure~\ref{fig:mergerradius}, the bulge fractions in Figure~\ref{fig:bulgeradius} vary much more significantly from the field value than the merger fraction. This suggests that minor mergers may be more dominant than major mergers for satellites in the group environment, particularly at $R/R_{200} > 0.5$. 

If mergers are a dominant evolutionary mechanism for satellite galaxies, as the morphological evidence in Figures~\ref{fig:mergerradius} \& \ref{fig:bulgeradius} suggests, we would expect to see a difference in the quenching histories of satellites residing in groups with a larger number of members. However, the bottom panels of Figure \ref{fig:timesinceradiusmu} show that there is no trend with time since quenching onset or rate of quenching with increasing $N_{group}$ for the satellite galaxies. This suggests that mergers are not the dominant quenching mechanism for satellite galaxies, but that whatever mechanism is the cause of the quenching occurs at the same rate irrespective of group size. 

Central galaxies however, do show a trend for increasing time since quenching with increasing $N_{group}$ (square points in Figure \ref{fig:timesinceradiusmu}c) occurring at a rate of $\tau \sim 1 \rm{Gyr}$. \cite{smethurst15} attributed these quenching rates to mergers and galaxy interactions which can transform a galaxy's morphology. Therefore, the larger the number of group members, the more likely a central galaxy has a history dominated by mergers. This is in agreement with the findings of \citet{lin10}, \citet{ellison10}, \citet{lidman13} and \citet{mcintosh08}. The latter found, by studying a sample of local groups and clusters, that half of the mergers they identified involved the central galaxy. \cite{liu09} also found that the fraction of merging centrals increases with the richness of a cluster (a measure of the number of galaxies within $1~\rm{h}^{-1}\rm{Mpc}$ of the central galaxy).

This idea is supported by the result in Figure~\ref{fig:timesinceradius}a showing that centrals of a given mass have quenched more recently than the inner satellites (at $\sim0.1R/R_{200}$) of a given mass. This suggests that an episode of more recent star formation, such as a starburst, may have occurred in the central galaxies but not in the inner satellites. Mergers are thought to cause an energetic burst of star formation which can in turn quench the remnant galaxy \citep{hopkins05, treister12, pontzen16}. This result is also suggestive of a merger dominated history for central galaxies but not for satellite galaxies.

\subsection{The role of mass quenching in the group environment}\label{sec:rolemassenv}

A trend is seen for increasing time since quenching with increasing stellar mass and velocity dispersion (a proxy for galaxy potential) for centrals, satellites and field galaxies in Figure~\ref{fig:timesinceradius}a and Figure~\ref{fig:timesinceradiusvel}c respectively. This is suggestive of mass quenching occurring across the entire galaxy population irrespective of environmental density, supporting the work of \citet{peng10, peng12, Gabor10} and \citet{darvish16}.

\subsection{The role of morphological quenching in the group environment}\label{sec:rolemorphenv}

The increasing bar fraction toward the central group regions shown in Figure \ref{fig:barradius} \citep[in agreement with][]{skibba12}, suggests that bars may be partly responsible for the relation between quenched fraction and environmental density. This is consistent with findings that show that bars themselves may be the cause of morphological quenching through the funnelling of gas toward the central regions of galaxies \citep{athanassoula92b, sheth05} which is then used in star formation, exhausting the available gas (see Section~\ref{sec:morphquench}).

%As discussed in Chapter~\ref{chap:agn}, an inflow due to a bar may also be able to fuel an AGN. The AGN-environmental connection has been extensively studied with conflicting results; \cite{pimbblet13}, with optically selected AGN, and \cite{elhert14}, with X-ray selected AGN, showed that the AGN fraction decreases towards the inner regions of groups. However, these studies did not take into account the morphology-density relation of satellite galaxies, where bulge dominated galaxies dominate in the inner regions of groups. \cite{miller03} and more recently \cite{desouza16}, using a hierarchical Bayesian method, showed with optically selected AGN that although the AGN fraction decreases with group radius for early-type galaxies, it stays constant with group radius for late-types. This suggests that disc galaxies are still able to fuel their AGN even in dense environments, perhaps due to the enhanced bar fraction in the inner regions of the group.

We must therefore consider whether the environment itself may play a role in triggering the disk instabilities which can produce a bar. Indeed harassment and tidal interactions, believed to be common in the group environment, have been shown to both promote and inhibit bar formation dependent on the stellar mass \citep{noguchi88, moore96, skibba12}.  If the environment was indeed triggering a bar, then morphological quenching would be occurring in the group environment but indirectly due to environmental quenching. This suggests that the polarity between internal secular processes (`nature') and external environmental processes (`nurture') may not be as extreme as first thought, in agreement with \cite{skibba12}. 

\subsection{The role of the environment in quenching}\label{sec:roleenv}

Across all panels of Figures~\ref{fig:timesinceradius}-\ref{fig:timesinceradiusvel} a trend for increasing time since quenching onset with decreasing group radius is present. We interpret this as environmentally driven mechanisms causing quenching at the same rate throughout the infall time of a galaxy in a group. Galaxies which are now closer in fell into the group earlier and as they did so they started to quench, giving rise to a larger inferred $\Delta t$.

More massive halos are seen to have a greater impact on the star formation histories of their satellites than less massive halos in Figure~\ref{fig:timesinceradius}c. The halo mass is correlated with both (i) the gravitational potential of the group and (ii) the temperature of the IGM, suggesting that an environmental quenching mechanism which is correlated with one or both of these properties is responsible for this result.

Higher mass halos have hotter intra group medium (IGM) temperatures \citep{shimizu03, delpopolo05} which can have a greater impact on a galaxy through ram pressure stripping (RPS) of cold gas. \cite{gunngott72} define the ram pressure as:
\begin{equation}\label{eq:rps}
\rho_{\rm{IGM}}\cdot v^2 = 2\pi G \cdot \sigma_*(R) \cdot \sigma_g(R),
\end{equation}
where $\rho_{\rm{IGM}}$ is the density of the IGM, $\sigma_*(R)$ the star surface density, $\sigma_g(R)$ the gas surface density of the galaxy disc and $v$ the velocity of the galaxy through the IGM. Therefore if RPS is indeed a dominant environmental quenching mechanism we should see a trend in $\Delta t$ with the velocity of a satellite relative to its central galaxy.  However in Figure \ref{fig:timesinceradiusvel}a we see that this is not the case. This therefore rules out RPS as the dominant environmental quenching mechanism, in support of the simulations of \citet{emerick16, fillingham16} which showed that RPS could only remove $40-60\%$ of a satellite's gas. However, this conclusion may be due to the stellar mass range spanned by the \textsc{gz2-group} satellite galaxies which all have $M_* \geq 10^9 \rm{M}_{\odot}$, as simulations by \cite{fillingham16} suggest that RPS only becomes effective in lower mass satellites with $M_* \leq 10^{8-9} \rm{M}_{\odot}$, in agreement with \cite{hester06}. 

Above this mass threshold in the simulations of \cite{fillingham16}, a `starvation' (or strangulation) mode \citep{larson80, balogh00} dominates, where a galaxy's extended gaseous halo is removed causing a quench, as cold gas for use in star formation can no longer be fed from the extended halo. This idea is supported by observations by \citet{peng10} which show that strangulation is a dominant mechanism for galaxies with $M_* < 10^{11}~M_{\odot}$ with a quenching timescale of $4~\rm{Gyr}.$ Such a mechanism will be correlated with the galaxy potential, as galaxies with a lower potential will be most easily stripped of their halos. This is apparent in Figure~\ref{fig:timesinceradiusvel}d where satellites with lower velocity dispersion (a proxy for the galaxy potential) are more rapidly quenched than their higher velocity dispersion counterparts and those in the field. Such a starvation mechanism is also correlated with halo mass, for which similar trends in $\Delta t$ are seen in Figure~\ref{fig:timesinceradius}c. The dominant environmental quenching mechanism occurring in the group environment must therefore be correlated with the group potential. This suggests that satellite galaxies may be most affected by gravitationally driven environmental effects, such as starvation, thermal evaporation of the galaxy halo and galaxy harassment. 


We can calculate an infall timescale for the satellite galaxies in the \textsc{gz2-group} sample if we assume that galaxies begin their infall into a group at a radius of $\sim10\rm{R}_{200}$ and stop infalling at $\sim0.1\rm{R}_{200}$\footnote{This assumes that galaxies will then merge with their central galaxy, however it is more likely that the satellite has a close pass with the central before it `backsplashes' into the group. See, for example, \cite{pimbblet11}.}. The difference in the time since quenching onset, $\Delta t$, between these two locations in a group will provide an estimate for how long it takes a satellite to infall. This assumes (i) that the galaxy starts to quench immediately when it enters the group and (ii) that the same environmentally driven quenching process is the only quenching mechanism affecting the satellites throughout their infall. We define this property as $\delta \Delta t = \Delta t_{0.1R_{200}} - \Delta t_{10 R_{200}}$. In Figure~\ref{fig:timesinceradiusmu}c the trend seen in $\Delta t$ with group radius is the same regardless of the number of galaxies in the group, so this gives us an estimate for the average $\Delta t$ in each group-centric radius bin across the satellite population. We therefore estimate an average infall time of $\delta \Delta t \sim 3 ~\rm{Gyr}$ for the \textsc{gz2-group} satellites. The rate of quenching occurring across the group radius in Figure~\ref{fig:timesinceradiusmu}d is $\tau \sim 1~\rm{Gyr}$ (within the range of quenching rates theorised to cause a morphological change by \citealt{smethurst15}) and so we can also estimate the average quenching timescale (i.e. the time taken to fully quench from the SFS to $5\sigma$ below the SFS) to be $\sim 4~\rm{Gyr}$ for the \textsc{gz2-group} satellites.

This infall time and quenching timescale are in agreement with the estimates of \cite{wetzel13} who used a high resolution cosmological N-body simulation to track satellite galaxy orbits in SDSS groups and clusters and found quenching timescales of $2-6~\rm{Gyr}$. Using a similar method, \cite{oman16} derive an infall time of $\sim4~\rm{Gyr}$ and quenching timescales between $4-6~\rm{Gyr}$ for galaxies in the mass range of the \textsc{gz2-group} sample. Similarly, \cite{hahn16} derive a total quenching timescale of $\sim4~\rm{Gyr}$ for satellite galaxies on infall into the group environment. However, the simulations by \cite{fillingham16} and \cite{emerick16} have shown that RPS cannot remove enough gas mass to completely quench a galaxy within $\sim2~\rm{Gyr}$ but can assist in reducing the starvation timescale so that galaxies can be quenched within the $\sim4~\rm{Gyr}$ quenching timescale calculated in this study. This suggests that although the effects of mechanisms correlating with the group potential are detectable in the quenching parameters of the \textsc{gz2-group} sample, this is only made possible by the constantly present, but less dominant effects of ram pressure stripping. 

This conclusion, along with that in Section~\ref{sec:rolemorphenv} where we noted that morphological quenching may only be present in the group environment due to the influence of the environment itself, suggests that all the mechanisms discussed here will affect a galaxy which is infalling through the group environment at some point in its lifetime. A single mechanism may be more dominant in the evolution of an individual galaxy but to achieve the correlations between morphology, colour and quenched galaxy fraction with density observed across the entire population, all mechanisms need to act in concert.  

\subsection{Link to other results using this method}\label{sec:bigpic}

Having considered the results presented here, we now consier these results in the context of previous results found using the \textsc{starpy} method presented in \cite{smethurst15} and \cite{smethurst16} for the entire \textsc{gz2-galex} sample and those hosting Type 2 AGN respectively.

% \cite{smethurst15} showed that quenching rates with $\tau < 1.5~\rm{Gyr}$ must be caused by mechanisms which can transform a galaxy from a disc to an elliptical. However this does not infer an immediate transition from a disc dominated to a bulge dominated galaxy. Work by the \textsc{ATLAS}$^{\rm{3D}}$ team \citep{cappellari11} showed the majority of the visually early-type population are rotationally supported \citep{emsellem11} with $\sim7$ times the number of \emph{fast rotators}, with kinematic discs, than \emph{slow rotators}, with dispersion dominated kinematics \citep[see][]{cappellari07, emsellem07}. Given the nature of visual morphology classifications, both fast and slow rotators will most likely have been classified as smooth galaxies and so both are expected to contribute to the smooth population SFHs of \cite{smethurst15}. 

% Dry major mergers are considered the most likely process to produce slow rotators \citep{duc11, naab14} as they can rapidly destroy the disc dominated nature of a galaxy \citep{toomre72}. \cite{smethurst15} find that $12\%$ of the red sequence smooth population have an inferred quenching rate $\tau <0.2~\rm{Gyr}$. Between $14-17\%$ of ellipticals are slow rotators \citep{emsellem11, stott16}, so this suggests that quenching mechanisms with these rates might give rise to a slow rotator (assuming the population densities are constant). This percentage is therefore also an estimate for the fraction of the galaxy population which have undergone a dry major merger, approximated by previous works to be $\sim10-20\%$ since $z\sim1$; \citep[][]{khochfar09}.

% If we now consider fast rotators, these are theorised to be formed by the slow build up of a galaxy's bulge over time, until it eventually overwhelms the disc. This growth is thought to occur via gas-rich major and minor mergers \citep{duc11} which can produce a bulge dominated, rotationally supported quenched galaxy, which would be visually classified as an elliptical. Although these mechanisms do not completely destroy the disc of a galaxy, they do cause an eventual morphological change to a visually bulge-dominated system. This could be the source of the result from \cite{smethurst15} suggesting that quenching mechanisms with $\tau < 1.5 ~\rm{Gyr}$ must be able to cause a morphological change. Although this morphological change does not produce a true dispersion dominated elliptical galaxy, the resulting galaxy will be identified as smooth by visual classifications. The large IFU studies of MaNGA \citep{bundy15}, SAMI \citep{croom12} and CALIFA \citep{sanchez12} will allow for larger populations of slow and fast rotators to be identified so that the relative dominance of gas-rich and dry mergers across the visually elliptical population can be determined more accurately. 

A parameter which is often investigated in quenching studies is the stellar mass surface density of a galaxy, which is found to correlate with SFR \citep{barro13b, whitaker16}. As a galaxy's bulge grows it is thought to be able to stabilise a disc against collapse and effectively stop it from forming stars. This is classed as a type of morphological quenching and is effective over time periods of a few $\rm{Gyr}$ \citep{Fang13} even if external gas is still fed to a galaxy. This slower quenching track of bulge dominated galaxies may help to explain the slow quenching rates observed by \cite{smethurst15} across the red and green smooth populations. They find that slow quenching with $\tau>2~\rm{Gyr}$ occurs for up to $40\%$ of the smooth green population and $24\%$ of the smooth red population. Using \textsc{starpy}, \cite{smethurst15} separated galaxies characterised by this slower quenching history, caused by processes which grow the bulge then consequently trigger morphological quenching, from those characterised by more rapid quenching histories, which are caused by processes which simultaneously quench the galaxy and grow the bulge. However, even in the latter case, morphological quenching may help in either speeding up the quenching process or in ensuring the galaxy stays quenched. This is supported by the finding of \cite{abramson16} who found that there is no threshold at which density triggered quenching occurs, but that denser systems redden faster than their less dense counterparts. This suggests that minor mergers and morphological quenching work together to fully achieve quiescence, similar to the collaboration between starvation and stripping to achieve quiescence of satellite galaxies discussed in Section~\ref{sec:roleenv}. 

This sort of partnership between two quenching mechanisms is also apparent in simulations which have shown that without AGN feedback a major merger cannot fully quench a galaxy \citep{springel05b}. In combination with a major merger however, a massive galaxy can be completely quenched by the heating or removal of gas and quiescence maintained \citep{conselice03, springel05b, hopkins08a, pontzen16}. These effects are therefore easily detectable, leading to the initial theories for the links between AGN and mergers \citep{merritt01, hopkins06b, hopkins08a, hopkins08b, peng07, jahnke11}. However, \cite{smethurst16} showed using \textsc{starpy} that galaxies hosting an AGN don't always quench at the rapid rates caused by major mergers, suggesting that a slow co-evolution of black hole and host galaxy can occur. They also showed that rapid quenching is only inferred for low mass AGN host galaxies where the AGN can have a greater impact on the galaxy SFR. 

Across the entire galaxy population we therefore have lots of examples of two quenching mechanisms working together to either quench a galaxy or ensure a galaxy stays quenched, including starvation and stripping (Section~\ref{sec:roleenv}), mergers \& AGN \citep{smethurst15, smethurst16}, disc instabilities \&  environment (Section~\ref{sec:rolemorphenv}) and minor mergers \& morphological quenching (Section~\ref{sec:bigpic}). All of these mechanisms result in the same end state of galaxy quiescence (with the occasional influx of gas thwarting their progress) but no single mechanism dominates over another, except in the most extreme environments or masses. While the effects of mass and morphological quenching are more apparent for galaxies in less dense environments (Figures~\ref{fig:timesinceradius}-\ref{fig:timesinceradiusvel}), they still affect galaxies in the densest environments. Similarly, the effects of mergers are much more apparent in galaxies in dense environments (e.g. centrals; see Section~\ref{sec:rolemergerenv}) and will often drown out the more subtle effects of slower quenching mechanisms which occurred before the merger. %The dominance of each mechanism is therefore a matter of circumstance. 

%I believe that it is the correct use of the morphological parameterisation that has allowed for all of these conclusions to be drawn. The evolution of a galaxy is continuous in nature from the most disc dominated to the most bulge dominated system. This nature is reflected by the continuous parameters which are used to describe this structure. This includes bulge-to-total ratios, S\'ersic index \citep{sersic68}, Gini coefficient \citep{abraham03, lotz04}, asymmetry \citep{conselice00} and concentration index \citep{morgan58}. A problem arises however, when studies discretise these values by mapping them to the typical distinct Hubble classifications of morphology; either the data is mapped to T-types \citep{shimasaku01, brinchmann04, nair10b, barro15} or merely split bimodally into late and early types, e.g. with either S\'ersic index, $n \leq 2.5$ \citep{ravindranath04, kelvin12, vika15} or GZ vote fraction, $p_d \geq 0.8$ \citep{schawinski14} to identify discs. This discretisation no longer reflects our uncertainty in the morphological classification due to the image resolution. With increasing redshift, galaxy structures can be washed out by the PSF of the image. This means that either large amounts of data must be discarded or the morphological bins made noisier by this uncertainty. By using the GZ vote fractions as weights in this study, this enabled me to retain all of the galaxies in my samples therefore utilising as much information as possible from across a galaxy population. This has allowed me to reveal the subtler effects of morphology and infer the broad range of quenching rates seen across the colour magnitude diagram. Treating the morphological classifications in this way, I was also able to reproduce the major differences between the populations seen in \cite{schawinski14} when a threshold on the GZ vote fractions was used.

Just as the morphology of galaxies is continuous in nature from disc to bulge dominated, so too are the effects of the quenching mechanisms which can cause this change. The impact of mergers on the morphology and SFR of a galaxy depends on the mass ratio, a continuous variable from micro mergers \citep{carlin16} through to major mergers. The strength of morphological quenching mechanisms can be measured on a continuum of stellar mass and stellar mass surface density of a galaxy; similarly the impact of environmentally driven quenching mechanisms increases with increasing halo mass. All of these processes, depending on a galaxy's environment, are likely to affect a galaxy at some point in its lifetime, acting in concert to reduce the SFR, which in turn produces the wide distribution of quenching timescales seen across the \textsc{gz2-group} sample. In previous works, efforts have been made to identify the dominant quenching mechanism in a galaxy sample \citep[e.g.][]{muzzin12, schawinski14, foltz15, woo15, balogh16, darvish16, huertascompany16}, yet it is clear from this study that multiple quenching mechanisms will affect galaxies across their lifetime, working in collaboration to ensure galaxies stay quenched. Future studies should therefore focus on disentangling the effects of these various different quenching mechanisms, rather than focussing on a single process. 

\section{Conclusions}\label{sec:conc}

Using the \citet{berlind06} group catalogue, we have constructed a sample of group galaxies in the SDSS which were cross matched with Galaxy Zoo 2 and GALEX in order to determine their most likely SFHs using \textsc{starpy}. We have shown that although mass quenching, morphological quenching and mergers are all important mechanisms at work in quenching the galaxies in the group environment, environmentally driven quenching mechanisms do play a role in quenching galaxies as they infall into the group. We have discussed the possibility that no single mechanism will dominate across the group population, with all mechanisms acting collaboratively. Our findings are summarised as follows:
\begin{enumerate}
\item The bar, obvious bulge and merger fractions are all seen to increase above the field value in the inner regions of the groups of the \textsc{gz2-group} sample in Figures~\ref{fig:barradius}, \ref{fig:bulgeradius} \& \ref{fig:mergerradius} respectively.  
 
\item Mergers are the dominant quenching mechanism for central galaxies but not for satellite galaxies. Satellites may undergo a minor merger in the group environment but their effects are only discernible by their indirect effect on the bulge fraction (see Figure~\ref{fig:bulgeradius}).
 
\item Mass dependent quenching is occurring across the entire \textsc{gz2-group} sample for both centrals and satellites irrespective of the environmental density (see Figure~\ref{fig:timesinceradius}a).
 
\item Morphological quenching is occurring for \textsc{gz2-group} satellite galaxies as evidenced by the heightened bar fraction in the inner group regions (see Figure~\ref{fig:barradius}). However, this may be indirectly due to environmental quenching since galaxy interactions and harassment are believed to be able to trigger bars. This suggests the polarity between `nature' vs. `nurture' may not be as extreme as previously thought, in agreement with \cite{skibba12}. 

\item The environment does cause quenching across the \textsc{gz2-group} sample, as evidenced by the increase in the time since quenching with decreasing group radius seen across all left panels of Figures~\ref{fig:timesinceradius}-\ref{fig:timesinceradiusvel}. The results in Figures \ref{fig:timesinceradius}a \&  \ref{fig:timesinceradiusvel}c suggest that this is caused by a quenching mechanism correlated with the group potential, such as harassment, interactions and starvation, rather than the velocity of a satellite through the group, such as ram pressure stripping. This quenching occurs within an average quenching timescale of $\sim4~\rm{Gyr}$ from star forming to complete quiescence, after an average infall time of $\sim 3 ~\rm{Gyr}$. 
  
\end{enumerate}

It is apparent from the results presented here that many quenching mechanisms are all occurring simultaneously in the group environment; therefore a superposition of all of the effects of these mechanisms is seen in the quenching histories of the \textsc{gz2-group} sample, which in turn gives rise to the observed morphology-density relation. 

\section*{Acknowledgements}

RS acknowledges funding from the Science and Technology Facilities Council Grant Code ST/K502236/1. 

The development of Galaxy Zoo was supported in part by the Alfred P. Sloan Foundation. Galaxy Zoo was supported by The Leverhulme Trust. 

Based on observations made with the NASA Galaxy Evolution Explorer.  GALEX is operated for NASA by the California Institute of Technology under NASA contract NAS5-98034

Funding for the SDSS and SDSS-II has been provided by the Alfred P. Sloan Foundation, the Participating Institutions, the National Science Foundation, the U.S. Department of Energy, the National Aeronautics and Space Administration, the Japanese Monbukagakusho, the Max Planck Society, and the Higher Education Funding Council for England. The SDSS Web Site is \url{http://www.sdss.org/}.
The SDSS is managed by the Astrophysical Research Consortium for the Participating Institutions. The Participating Institutions are the American Museum of Natural History, Astrophysical Institute Potsdam, University of Basel, University of Cambridge, Case Western Reserve University, University of Chicago, Drexel University, Fermilab, the Institute for Advanced Study, the Japan Participation Group, Johns Hopkins University, the Joint Institute for Nuclear Astrophysics, the Kavli Institute for Particle Astrophysics and Cosmology, the Korean Scientist Group, the Chinese Academy of Sciences (LAMOST), Los Alamos National Laboratory, the Max-Planck-Institute for Astronomy (MPIA), the Max-Planck-Institute for Astrophysics (MPA), New Mexico State University, Ohio State University, University of Pittsburgh, University of Portsmouth, Princeton University, the United States Naval Observatory, and the University of Washington.

This publication made extensive use of the Tool for Operations on Catalogues And Tables (TOPCAT; ~\citealt{taylor05}) which can be found at \url{http://www.star.bris.ac.uk/~mbt/topcat/} and the open source Python module \emph{astroPy}\footnote{\url{http://www.astropy.org/}}; \citealt{astropy13}). This research has also made use of NASA's ADS service and Cornell's ArXiv. 


\bibliographystyle{mn2e}
\bibliography{refs}  

\end{document}
