\documentclass[useAMS,usenatbib]{mn2e}
\usepackage{footnote,graphicx,natbib,color,multirow,amsmath,url}
\voffset=-0.5in
\pdfminorversion=5
\setlength{\parsep}{0pt}
\setlength{\partopsep}{0pt}

\def\starpy ~{\textsc{starpy}}

\font\nbf=cmssbx10 at 12.28pt %big font for headers


\begin{document}


\title[Environmental quenching: yay or nay?]{Galaxy Zoo: Quenching timescales of group galaxies}
\author[Smethurst et al. 2015]{R. ~J. ~Smethurst,$^{1}$ C. ~J. ~Lintott,$^{1}$ and the Galaxy Zoo team \footnotemark[1]
\\ $^1$ Oxford Astrophysics, Department of Physics, University of Oxford, Denys Wilkinson Building, Keble Road, Oxford, OX1 3RH, UK 
}

\maketitle

\begin{abstract}
Does the environment of a galaxy directly influence the quenching timescale of a galaxy? Here we construct a sample of group and field galaxies with morphological classifications from Galaxy Zoo 2 and use Bayesian inference to determine the quenching time and exponential timescale that describes a simple SFH for a given galaxy from its photometry. We observe how the detailed morphological structures, such as bars and bulges, are affected by environment and correlate this with changes in the quenching timescales. Mass quenching is seen to be just as important for satellite galaxies as central galaxies.  We find that the environment does play a direct role in galaxy quenching through a mechanism which is correlated to the halo mass of a group. It is not correlated with the relative velocity of satellite galaxies within the group, therefore mechanisms such as ram pressure stripping are not the main cause of quenching in the group environment. 

\end{abstract}

\begin{keywords}
galaxies -- environment, galaxies -- photometry, galaxies -- statistics, galaxies -- morphology
\end{keywords}

\footnotetext[1]{This investigation has been made possible by the participation of over 350,000 users in the Galaxy Zoo project. Their contributions are acknowledged at \url{http://authors.galaxyzoo.org}}

\section{Introduction}\label{sec:intro}
 
 There are many mechanisms which are proposed to cause quenching; including mergers \citep{daddi10}, mass quenching \citep{kennicutt77, peng12}, morphological quenching \citep{faber12} and the environment of a galaxy.
 
 The galaxy environment as a cause of quenching was proposed due to the correlation of both morphology \citep{dressler80} and the quenched galaxy fraction \citep{?} with environmental density. However, does this correlation truly imply causation? Evidence from simulations \citep{?} suggests that the environment may not be the dominant quenching mechanism in the galaxy lifecycle. Perhaps the correlation of increased galaxy quenched fractions with environment is due to a superposition of the effects of mergers, interactions and both mass \& morphology quenching. In denser environments, galaxies are more likely to encounter another galaxy in a merger or interaction scenario and infall over long timescales during which gas reservoirs can be depleted due to star formation via mass quenching.
 
There has been mounting evidence that the oft proposed mechanism behind environmental quenching - ram pressure stripping \citep{?} - may not be effective at quenching as first thought \citep{}. 
 
In order to isolate the case of the density-morphology and density-SFR correlations, we need to study how galaxy quenching timescales and morphology change in group and field environments with different properties. To do this we construct a sample of both group and field galaxies and utilise Bayesian inference to determine the quenching time and exponential timescale to describe a simple SFH for a galaxy given its photometry. We aim to determine the following: (i) How does the environment influence the detailed morphological structures of a galaxy?  (ii) Does the environment directly cause quenching? (iii) If so, what environmental mechanism is responsible for this quenching?

In Section~\ref{sec:data} we describe our data sources and inference methods and highlight our results in Section~\ref{sec:results}. We then discuss these results and how they fit into the bigger picture of quenching in Section~\ref{sec:disc}. The zero points of all magnitudes are in the AB system. Where necessary, we adopt the WMAP Seven-Year Cosmology \citep{jarosik11} with $(\Omega_m , ~\Omega_\Lambda , ~h) = (0.26, 0.73, 0.71)$.

 
\section{Data and Methods}\label{sec:data}

\subsection{Data Sources}\label{sec:photo}

In this investigation we use visual classifications of galaxy morphologies from the Galaxy Zoo 2\footnote{\url{http://zoo2.galaxyzoo.org/}} (GZ2) citizen science project \citep{GZ2}, which obtains multiple independent classifications for each optical image. The full question tree for an image is shown in Figure~1 of \citeauthor{GZ2}  The GZ2 project used $304, 022$ images from the Sloan Digital Sky Survey Data Release 7 (SDSS; \citealt{york00, abazajian09}) all classified by \emph{at least} 17 independent users, with a mean number of classifications of $\sim42$.

Further to this, we required NUV photometry from the GALEX survey \citep{martin05}, within which $\sim42\%$ of the GZ2 sample was observed, giving $126, 316$ galaxies total ($0.01 < z < 0.25$). This will be referred to as the \textsc{gz2-galex} sample. The completeness of this sample ($-22 < M_u < -15$) is shown in Figure~2 of \cite{smethurst15}. 

Observed fluxes are corrected for galactic extinction \citep{Oh11} by applying the \citet{cardelli89} law. We also adopt $k$-corrections to $z = 0.0$ and obtain absolute magnitudes from the NYU-VAGC \citep{blanton05, blanton07, padmanabhan08}.

\begin{figure*}
\centering{
\includegraphics[width=0.95\textwidth]{mosaic_sat_cent_field_disc_fraction.pdf}
\caption{Randomly selected SDSS \emph{gri} composite images of satellite and central galaxies in the \textsc{gz2-group} sample in comparison to those from the \textsc{gz2-field} sample. All galaxies lie within $0.04 < z < 0.05$ and in the central galaxy mass range $10^{10.5} < M_{*} [M_{\odot}] < 10^{11}$, used as a proxy for halo mass. The galaxies are ordered from least to most featured according to their debiased `disc or featured' vote fraction, $p_d$ (see \citealt{willett13}). The scale for each image is $0.099~\rm{arcsec/pixel}$.}}
\label{fig:mosaic}
\end{figure*}

\subsubsection{Group Identification}\label{sec:groups}



We used the \citet{berlind06} catalogue, which uses a friends-of-friends algorithm to identify group and cluster galaxies in the SDSS. This was cross matched to the \textsc{gz2-galex} sample and limited to $z < 0.1$ to ensure GALEX completeness of the red sequence (see \citealt{wyder07, yesuf14}). Centrals were selected as the most massive galaxy in a group and all others were designated as satellites (masses were calculated using the absolute r-band magnitude and $u-r$ colour and the method of \citealt{baldry06}).  The projected cluster centric radius, $R$, of all satellite galaxies was calculated from the projected separations of a satellite from it's central (converted to $\rm{kpc}$ from a consideration of the observed redshift of the central galaxy) which was then normalised by the $R_{200}$ of the group (the radius within which the mass overdensity is 200 times the critical density; see \citealt{?}), often used as a proxy for the virial radius of the group \citep{?}. This resulted in a sample of $14,199$ group galaxies with $3,468$ centrals and $10,731$ satellites within a projected cluster centric radius range of $0 < R/R_{200} < 25$ and $z < 0.084$. 

In this work we focus on galaxies which are either quenching or quenched and are more than $\pm1\sigma$ below the star forming `main sequence'. This encompasses $4629$ satellite and $2314$ central galaxies and will be referred to as the \textsc{gz2-group} sample. These galaxies are highlighted in red on Figure \ref{}. 

\subsubsection{Field sample}\label{sec:field}

For all galaxies in the \textsc{gz2-galex} sample, we calculated the smallest projected cluster centric radii, $R/R_{200}$, from each of the central galaxies in the  \citet{berlind06} catalog and selected candidate field galaxies as those with (i) $R/R_{200} > 25$ and (ii) $\log\Sigma < -0.8$ \citep[a measure of environmental density from][]{baldry06}. This sample of field galaxy candidates was then matched in redshift and stellar mass firstly to the central galaxies of the \textsc{gz2-group} sample to give $2,309$ field galaxies with $z < 0.084$ which will be referred to as the \textsc{gz2-cent-field} sample. In this work we focus on galaxies which are either quenching or quenched and are more than $\pm1\sigma$ below the star forming `main sequence'. This encompasses $1,596$ field galaxies with $z < 0.084$ which will be referred to as the \textsc{gz2-cent-field-q} sample and used as a control sample when investigating the trends with central galaxy properties of the inferred quenching parameters. %KS test between samples?


Secondly, the field galaxy candidates were then matched in redshift and stellar mass to the satellite galaxies of the \textsc{gz2-group} sample to give $6,849$ field galaxies with $z < 0.084$ which will be referred to as the \textsc{gz2-sat-field}. These galaxies in the \textsc{gz2-sat-field} sample will be used as a control when investigating the morphological trends of satellite galaxies with environment. Note that the sample is not restricted to being $\pm1\sigma$ below the star forming `main sequence' in this case. %KS test between samples?

%We also select all those galaxies in the satellite matched sample $\pm1\sigma$ below the star forming `main sequence', giving $1,043$ quenching or quenched field galaxies which will be referred to as the \textsc{gz2-sat-field-q} sample. These galaxies in the \textsc{gz2-sat-field-q} sample will be used as a control when investigating the trends with environment of the inferred quenching parameters. 
  

\begin{figure}
\centering{
\includegraphics[width=0.45\textwidth]{redshift_cent_field.pdf}
\caption{Redshift distributions of central galaxies in the \textsc{gz2-group} sample (black solid line) in comparison to the redshift and mass matched \textsc{gz2-cent-field-q} sample (blue dashed line).}}
%KS Test between distributions?
\label{fig:zcompare}
\end{figure}

\subsection{Deriving quenching parameters}\label{sec:starpy}

\textsc{starpy}\footnote{Publicly available: \url{http://github.com/zooniverse/starpy}} is a \textsc{python} code which allows the user to derive the quenching star formation history (SFH) of a single galaxy through a Bayesian Markov Chain Monte Carlo method \citep{emcee13}\footnote{\url{http://dan.iel.fm/emcee/}} with the input of the observed $u-r$ and $NUV-u$ colours, a redshift, and the use of the stellar population models of \cite{BC03}.  These models are implemented using solar metallicity (varying this does not substantially affect these results; \citealt{smethurst15}) and a Chabrier IMF \citep{Chab03} but does not model for intrinsic dust. The SFH is modelled as an exponential decline of the SFR described by two parameters $[t_{q}, \tau]$, where $t_{q}$ is the time at the onset of quenching $\rm{[Gyr]}$ and $\tau$ is the exponential rate at which quenching occurs $\rm{[Gyr]}$. Under the simplifying assumption that all galaxies formed at $t=0$ $\rm{ Gyr}$ with an initial burst of star formation, the SFH can be described as: 
\begin{equation}\label{sfh}
SFR =
\begin{cases}
i_{sfr}(t_{q}) & \text{if } t < t_{q} \\
i_{sfr}(t_{q}) \times exp{\left( \frac{-(t-t_{q})}{\tau}\right)} & \text{if } t > t_{q} 
\end{cases}
\end{equation}
where $i_{sfr}$ is an initial constant star formation rate dependent on $t_{q}$ \citep{schawinski14, smethurst15}.  A smaller $\tau$ value corresponds to a rapid quench, whereas a larger $\tau$ value corresponds to a slower quench. We note that a galaxy undergoing a slow quench is not necessarily quiescent by the time of observation. Similarly, despite a rapid quenching rate, star formation in a galaxy may still be ongoing at very low rates, rather than being fully quenched. This SFH model has previously been shown to appropriately characterise quenching galaxies \citep{Weiner06, Martin07, Noeske07,schawinski14}. We note also that star forming galaxies in this regime are fit by a constant SFR with a $t_{q} \simeq$ Age$(z)$, (i.e. the age of the Universe at the galaxy's observed redshift) with a very low probability.

The probabilistic fitting methods to these star formation histories for an observed galaxy are described in full detail in Section 3.2 of \cite{smethurst15}, wherein the \starpy ~~code was used to characterise the SFHs of each galaxy in the \textsc{gz2-galex} sample. We assume a flat prior on all the model parameters and the difference between the observed and predicted $u-r$ and $NUV-u$ colours are modelled as independent realisations of a double Gaussian likelihood function (Equation 2 in \citealt{smethurst15}). We also make the simplifying assumption that the age of each galaxy, $t_\mathrm{age}$ corresponds to the age of the Universe at its observed redshift, $t_\mathrm{obs}$.

We note also that galaxy colours were not corrected for intrinsic dust attenuation. This is of particular consequence for disc galaxies, where attenuation increases with increasing inclination. \cite{Buat05} found the median value of the attenuation in the GALEX NUV passband to be $\sim 1$ mag. Similarly \cite{masters10c} found a total extinction from face-on to edge-on spirals of 0.7 and 0.5 mag for the SDSS $u$ and $r$ passbands and show spirals with $\log(a/b) > 0.7$ have signs of significant dust attenuation. However, we showed from an investigation into this problem in Section~2.2 of \citet{smethurst16} that internal galactic extinction does not systematically bias our results from \starpy~. 

The output of \starpy  ~ is probabilistic in nature and provides the posterior probability distribution across the two-parameter space for an individual galaxy the degeneracies for which can be seen in Figure~4 of \citet{smethurst15}.  Whereas in previous studies that distributions have been weight and combined to give an overall distribution representing the population of galaxies \citep[see][]{smethurst15, smethurst16}, due to the complex nature of the group environment here we choose instead to take the 50th percentile walker position of these individual posterior probability distributions to give the most likely $t_{q}$ and $\tau$ for each galaxy. This simplifies the output from \starpy ~for each galaxy from a probability distribution to two more manageable values. In this work we will look for trends in the time since quenching onset, $\Delta t$, for a given galaxy by calculating $\Delta t = t_\mathrm{obs} - t_{q}$. We will observe how this quantity changes with group properties, including halo mass (here we use the stellar mass of the group central as a proxy for halo mass), velocity dispersion, number of group members and relative velocity of a satellite galaxy. 

\section{Results}\label{sec:results}

\begin{figure}
\includegraphics[width=0.46\textwidth]{p_disc_trend_with_log_radius_field_compare.pdf}
\includegraphics[width=0.46\textwidth]{p_smooth_trend_with_log_radius_field_compare.pdf}
\caption{Mean GZ vote fraction for disc (top) and smooth (bottom) galaxies in the \textsc{gz2-group} sample binned in projected cluster centric radius, normalised by $R_{200}$, a proxy for the virial radius of a group. The shaded region shows $\pm1\sigma$ on the mean vote fraction. The mean vote fraction of the \textsc{gz2-sat-field} sample are also shown (blue solid lines) with $\pm1\sigma$ (blue dashed lines).}
\label{fig:morphradius}
\end{figure}

\begin{figure}
\includegraphics[width=0.46\textwidth]{bar_fraction_over_disc_trend_with_log_radius_sat_matched_field_cand.pdf}
\caption{Bar fraction (number of barred disc galaxies over number of disc galaxies) in the \textsc{gz2-group} sample binned in projected cluster centric radius, normalised by $R_{200}$, a proxy for the virial radius of a group. The shaded region shows $\pm1\sigma$ on the bar fraction. The bar fraction of the \textsc{gz2-sat-field} sample is also shown (blue solid line) with $\pm1\sigma$ (blue dashed line).}
\label{fig:barradius}
\end{figure}

\begin{figure}
\includegraphics[width=0.46\textwidth]{merger_fraction_trend_with_log_radius_compare_sat_field_cand.pdf}
\caption{Merger fraction in the \textsc{gz2-group} sample binned in projected cluster centric radius, normalised by $R_{200}$, a proxy for the virial radius of a group. The shaded region shows $\pm1\sigma$ on the merger fraction. The merger fraction of the \textsc{gz2-sat-field} sample is also shown (blue solid line) with $\pm1\sigma$ (blue dashed line).}
\label{fig:mergerradius}
\end{figure}

The extensive morphological classifications provided by GZ2 allows us not only to sanity check the \textsc{gz2-group} sample for the typical morphology-density relation \citep{dressler80} but also investigate how more detailed structure is affected by the group environment.  Figure \ref{fig:morphradius} shows the mean disc and smooth vote fractions from galaxy zoo, binned in projected cluster centric radius (normalised by the approximate viral radius of each group, $R_{200}$). We can see that the mean disc (smooth) vote fraction decreases (increases) from the mean field value (blue line) under $1$ virial radius.

Figure \ref{fig:barradius} shows how the bar fraction (number of barred disc galaxies over the number of disc galaxies) increases towards the centre of the group population, significantly over the field fraction (blue solid line). This suggests the that the environment may play a role in triggering the disk instabilities which produce a morphological bar \citep{ref, ref, ref}. This is consistent with the results that show that bars themselves may be the cause of morphological quenching through the funnelling of gas toward the central regions of galaxies \citep{} but implies that therefore implies that the quenching observed in the group population may not be due directly to an environmental quenching mechanism, but indirectly by the triggering of a bar. 

Figure \ref{fig:mergerradius} shows how the merger fraction does not significantly deviate from the field fraction (blue solid line) until within $\sim$ virial radius. The merger classification in GZ2 has been shown to preferentially identify major mergers \citep{}. Similarly, bulge formation in disc galaxies is often associated with (minor) merger driven evolutionary histories \ref{}. The left panel of Figure \ref{fig:bulgeradius} shows how those galaxies identified as having no or just noticeable bulges are less common in the inner regions of the cluster (left panel), whereas the fraction of galaxies with obvious or dominant bulges increases with decreasing projected distance from the centre of the cluster. The bulge fractions in Figure 
~\ref{fig:bulgeradius} vary much more significantly from the field value than the merger fraction in Figure~\ref{fig:mergerradius}. This suggests that minor mergers may be more dominant than major mergers in the group environment, particularly beyond $0.5 R/R_{200}$. 

Figure \ref{fig:sfrradius} shows how the SFR of the \textsc{gz2-group} sample does indeed decline with decreasing cluster centric distance, significantly below the mean SFR of the \textsc{gz2-field} sample shown by the blue dashed line. This is in agreement with the results of \cite{gomez03} who observe a similar decline in SFR with cluster centric radius in SDSS clusters (see for example, Figure 6 in \citealt{gomez03}). This coincides with the morphological fraction changes seen in Figures~\ref{fig:morphradius}-{\ref{fig:merger radius} in support of the conclusions of \citet{smethurst15} that quenching is morphologically dependant. 

\begin{figure*}
\includegraphics[width=0.85\textwidth]{min_max_bulge_fraction_trend_with_log_radius_sat_field_cand.pdf}
\caption{Fraction of galaxies with none/just noticeable bulge classifications (left) and with obvious/dominant bulge classifications (right) in the \textsc{gz2-group} sample binned in projected cluster centric radius, normalised by $R_{200}$, a proxy for the virial radius of a group. The shaded regions shows $\pm1\sigma$ on the bulge fractions. The bulge fractions of the \textsc{gz2-sat-field} sample are also shown (blue solid lines) with $\pm1\sigma$ (blue dashed lines).}
\label{fig:bulgeradius}
\end{figure*}

\begin{figure}
\includegraphics[width=0.46\textwidth]{sfr_trend_with_log_radius_field_matched_blue_dashed_hlines_gomez_03_rv_not_r200.pdf}
\caption{Median $H\alpha$ derived star formation rates of satellite galaxies in the \textsc{gz2-group} sample, binned in projected cluster centric radius, normalised by $R_{200}$, a proxy for the virial radius of a group.  The shaded region shows the SFRs encompassed by $50\%$ of the population in a given bin. The median SFR of the \textsc{gz2-sat-field} sample is shown (blue solid line) along with the 25th and 75th percentiles (blue dashed lines).}
\label{fig:sfrradius}
\end{figure}


With the output from \starpy~ we can also study the time since quenching onset ($\Delta t = t_{obs} - t_{q}$, see Section \ref{}) binned in projected cluster centric radius, normalised by $R_{200}$ (a proxy for virial radius) for satellite galaxies and central galaxies in the \textsc{gz2-group} sample, compared with galaxies in the \textsc{gz2-field} sample. We can investigate these trends with group properties as shown in Figures \ref{fig:timesinceradius} \& \ref{fig:timesinceradiusvel}. 

Across all the panels in Figures \ref{fig:timesinceradius} \& \ref{fig:timesinceradiusvel} we see a general trend for increasing time since quenching onset with decreasing distance from the group centre, which is suggestive that this trend is due to an environmental quenching mechanism. As earlier, in Figures \ref{fig:morphradius}$-$\ref{fig:bulgeradius} significant differences from the field value arise inside $\sim$ one virial radius. 

If mergers are an important evolutionary mechanism for satellite galaxies as the morphological evidence in Figures~\ref{fig:mergerradius} \& \ref{fig:bulgeradius} suggests, we would expect to see a difference in the quenching histories of satellites in groups with a higher number of galaxies, $N_{group}$. However, the bottom panel of Figure \ref{fig:timesinceradius} shows that there is no trend with time since quenching onset with increasing $N_{group}$ for the satellite galaxies. The central galaxies (shown by the square points at $\sim 0.01 R/R_{200}$) however, do show a trend for increasing time since quenching with $N_group$. Suggesting that mergers are not the dominant quenching mechanism for satellite galaxies but are for centrals.  

In the middle panel of Figure \ref{fig:timesinceradius} the satellite galaxies of the \textsc{gz2-group} sample are now split by their stellar mass (calculated from the absolute r-band magnitude and $u-r$ colour by the method outlined in \citealt{baldry06}) and we do see a clear trend for increasing time since quenching onset with increasing stellar mass for both satellite and central galaxies. This is suggestive of mass quenching among the group galaxy population. This is contrary to previous work suggesting that mass quenching is only of import for central galaxies \citep{ref, ref, ref}. Interestingly, the inner satellites of a given mass have quenched less recently than the centrals at the same mass range, suggesting some episode of more recent star formation may have occurred in the central galaxies but not in the inner satellites. This is once again suggestive of a merger dominated evolutionary history for central galaxies, with mergers postulated to cause a burst of star formation before then quenching the remnant galaxy \citep{?,?, pontzen16}. 

In simulations, the three things that are found to most constrain galaxy evolution are redshift, mass and halo mass \cite{ref, ref}. To study the effect that halo mass has on the quenching properties of group galaxies we shall use a proxy for halo mass by splitting by the \textsc{gz2-group} sample by the stellar mass of the corresponding central galaxy of a group.

This is shown in the top panel of Figure \ref{fig:timesinceradius} where we can once again see a clear trend for increasing time since quenching onset with increasing stellar mass of the group central for both satellite and central galaxies. More massive halos therefore have a greater impact on the star formation histories of their satellites than less massive halos. This is often though to be attributed to hotter inter galactic medium (IGM) temperatures in higher mass halos which can then impact on a galaxy through ram pressure stripping (RPS) of gas for star formation. If RPS is indeed a dominant environmental quenching mechanism we should therefore see a trend in $\Delta t$ with the speed of a satellite galaxy relative to the group central.  In the bottom panel of Figure \ref{fig:timesinceradiusvel} we split the satellite galaxies of the \textsc{gz2-group} sample into bins of relative velocity to their central galaxies. We can see that there is no trend with time since onset of quenching with increasing relative velocity for satellite galaxies, however the trend with decreasing projected group centric radius, seen in each panel in Figure \ref{fig:timesinceradius} is still present. This suggests that any environmental processes causing this quenching are not corrected with satellite velocity and therefore RPS is not the dominant environmental quenching mechanism, in support of the conclusions of \citep{?}.

We can also account for both the stellar mass and the halo mass of the central galaxy simultaneously by considering the stellar mass ratio of the satellite to it's central galaxy, $\mu_* = M_*/M_{*,c}$, once again using the stellar mass of the central galaxy as a proxy for halo mass. In the middle panel of Figure \ref{fig:timesinceradiusvel} we show the time since quenching of the \textsc{gz2-group} sample with projected cluster centric radius split into bins of $\mu_*$. The change in $\Delta t $ with projected cluster centric radius occurs more steeply (particularly beyond $\sim$ a virial radius) for satellite galaxies with much smaller masses than their group central ($-2.0 < \log_{10}\mu_* < -1.0$, shown by the blue curve). Since the stellar mass of the central galaxy is correlated with the halo mass and therefore the potential of the system, this suggests that smaller mass galaxies in larger halos are most effected by environmental effects, therefore the dominant environmental quenching mechanism must be correlated with the group potential. 

Previous studies have claimed that the property which correlates most with whether a galaxy is quenched is the stellar velocity dispersion, $\sigma_*$. Shown in the top panel of Figure \ref{fig:timesinceradiusvel} is the time since quenching of the \textsc{gz2-group} sample with projected cluster centric radius split into bins of $\sigma_*$. Along with the stellar mass (shown in the middle panel of Figure \ref{fig:timesinceradius}), the stellar velocity dispersion shows the largest trend in $\Delta t$  for satellite galaxies, with galaxies with the smallest (largest) stellar velocity dispersions have quenched more (less) recently. 

\begin{figure}
\centering{
\includegraphics[height=0.875\textheight]{time_since_quenching_environment_properties.pdf}
\caption{The time since quenching onset ($\Delta t = t_{obs} - t_{q}$) binned in projected cluster centric radius, normalised by $R_{200}$, for satellite galaxies (triangles) split by stellar mass of the corresponding central galaxy (top), stellar mass (middle) and the number of galaxies within the group (bottom). The corresponding values for central galaxies (squares) and galaxies in the \textsc{gz2-cent-field-q} sample (circles) are shown and connected by the dashed lines to aid the reader.}
\label{fig:timesinceradius}}
\end{figure}

\begin{figure}
\centering{
\includegraphics[height=0.875\textheight]{time_since_quenching_v_disp_mu_bins_delv.pdf}
\caption{The time since quenching onset ($\Delta t = t_{obs} - t_{q}$) binned in projected cluster centric radius, normalised by $R_{200}$, for satellite galaxies (triangles) split by velocity dispersion (top), stellar mass ratio ($\mu_* = M_*/M_{*,c}$) (middle) and the difference in velocity from the associated central galaxy (bottom). The corresponding values for central galaxies (squares) and galaxies in the \textsc{gz2-cent-field-q} sample (circles) are shown and connected by the dashed lines to aid the reader in the top panel where appropriate.}
\label{fig:timesinceradiusvel}}
\end{figure}


\section{Discussion}\label{sec:disc}

\subsection{The role of environment in quenching}\label{sec:roleenv}

Across all panels of Figures ~\ref{fig:timesinceradius}-\ref{fig:timesinceradiusvel} a trend for increasing time since quenching onset with decreasing projected cluster centric radius was present. This suggests that the environment does directly cause quenching; galaxies closer in, fell into the group earlier and as they did so they started to quench giving rise to a larger $\Delta t$. However, as seen in Figure \ref{fig:timesinceradiusvel} there is no trend in the time since quenching onset with the relative velocity of the satellites to their corresponding central and a steeper trend with $R/R_{200}$ for lower mass satellites in larger mass halos. This suggests that whatever environmental mechanism is at play here, it is dependant on the size of the halo, either due to the potential or temperature of the halo, but not dependant on the speed of the satellite. This suggests that ram pressure stripping is not the dominant environmental quenching mechanism at play. 

We have shown that mergers are important for centrals not for satellites in the bottom panel of Figure \ref{fig:timesinceradius}. That mass quenching is important for satellites as well as centrals in the middle panel of Figure \ref{fig:timesinceradius} and that larger halos have a stronger environmental effect on their satellites in the top panel of Figure \ref{fig:timesinceradius}. 


\subsection{The Big Picture}\label{sec:bigpic}

In Section \ref{sec:results} we have discussed how our results of the changing morphological features and quenching timescales with projected cluster centric radius in the group environment show evidence for both merger driven, secular and environmentally driven evolutionary histories. This suggests that not one mechanism is dominant in the group environment but that a superposition of all these effects gives rise to the observed morphology-density and morphology-SFR relations.

All these mechanisms are striving towards the same end result with no single mechanisms dominating over the other. Those mechanisms traditionally associated with the field, such as secular evolution can also occur in more dense environments, however will often eventually be overwhelmed by those more rapid and violent mechanisms of mergers and interactions (and the triggered outflows from AGN that are associated with such mechanisms; see \citealt{smethurst16}). Similarly, the environmental quenching mechanisms are at work as soon as a galaxy falls into a group or cluster, but such a process can be interrupted momentarily by an interaction or a merger as a galaxy enters the more dense environment.   

Just as morphology is a spectrum from disc-dominated to spheroid-dominated systems, so to are the quenching mechanisms which cause this morphological transformation. Mergers and interactions are a spectrum of mass ratios from the micro mergers \citep{?} through to major mergers, with increasing impact upon the morphology and SFR of a galaxy. Secular quenching mechanisms are a spectrum of stellar mass, with a larger impact on those galaxies with smaller masses. Environmental quenching mechanisms are a spectrum of increasing halo potential, giving rise to a stronger impact on the SFR of smaller mass galaxies in larger halos.  

All of these mechanisms coalesce will give rise to the distributions in galaxy properties we see across the Universe through their constant interplay across cosmic time. 


\section{Conclusions}\label{sec:conc}

Mass quenching is definitely prevalent for satellites and mergers are important only in the most inner regions of groups for central galaxies. The environment does play a role in quenching galaxies through a mechanism proportional to the halo mass of the group but which isn't proportional to the speed the satellite galaxy is moving at relative to it's central galaxy. This suggests that ram pressure stripping is therefore not the dominant mechanism at work in environmental quenching. 

Everything in cahoots. 

\bibliographystyle{mn2e}
\bibliography{refs}  

\end{document}